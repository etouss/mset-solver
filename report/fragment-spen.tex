%!TEX root = main.tex

\section{Multi-set Constraints in \spen}
\label{sec:fspen}

The theory of data used in the \spen\ solver combines the quantifier free theory of integers and quantifier free theory of multi-sets over integers.
The comparison operation over integers is extended to multi-sets in the natural way
($M_1\leq M_2$ iff any element of $M_1$ is less or equal to any element of $M_2$).
The decision problems for this theory (satisfiability and entailment) have been shown decidable by reduction to the theory of integers with uninterpreted functions in~\cite{???} 
Moreover, these results have been extended to the aforementioned theory with size constraints in~\cite{DBLP:conf/cade/PiskacK10}.
We denote by $c$ the integer constants, 
by $v,v_1,v_2,\ldots$ the integer variables (with values in $ZZ$), and 
by $M,M_1,M_2,\ldots$ the multiset variables (with values in $\ZZ\rightarrow \NN$). 

The theory of data used in \spen\ is defined by the following grammar:
\begin{align*}
e &::= 
c ~\mid~     % constant
v ~\mid~     % variable
-e ~\mid~    % unary minus 
e + e ~\mid~ % unary plus
e - e ~\mid~ % binary minus
\ite(\varphi,e,e)
& \mbox{integer expressions} 
\\[2mm]
b &::= 
\emptyset ~\mid~     % empty bag
\bsingle(v) ~\mid~   % singleton bag
M ~\mid~             % bag variable
b \bplus b  ~\mid~   % bag union
b \bminus b ~\mid~   % bag minus
\ite(\varphi,b,b)
& \mbox{multiset expressions}
\\[2mm]
\# &::=\ = ~\mid~ \neq ~\mid~ < ~\mid~ \leq ~\mid~ > ~\mid~ \geq 
& \mbox{comparison operators} 
\\[2mm]
\varphi &::=
true ~\mid~ 
e \# e ~\mid~ % comparison of integers
b \# b ~\mid~ % comparison of bag
b \subset b ~\mid~ % subset of bags
\varphi \land \varphi ~\mid~
\varphi \limp \varphi 
& \mbox{formulas}
%\\[2mm]
\end{align*}

The translation to the SMTLIB2 theory AUFLIA is provided in~\cite{DBLP:conf/sigsoft/KapurMZ06}.


