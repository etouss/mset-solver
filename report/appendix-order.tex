%!TEX root = paper.tex

\section{Decision Procedure by Reduction to \textsf{QFUFLIA}}
\label{app:forder}

\subsection{QFBILIAUF Semantics}
\label{ssec:semUF}
A valuation $I_{\textit{uf}}$ of variables in $V_{\textit{uf}}$ is a function mapping variables in $V_{\textit{uf}}$ to values in $(\ZZ \rightarrow \ZZ)$, i.e.,
$I_{\textit{uf}} \; : \; V_{\textit{uf}} \flc (\ZZ \rightarrow \ZZ)$.
A valuation $I$ of variables in $V_{int}\cup V_{\textit{uf}} \cup V_{bag} \cup V_{ext}$ is a tuple of valuations $(I_{int}, I_{\textit{uf}}, I_{bag}, I_{ext})$;
we denote by $I_{int}$, $I_{\textit{uf}}$, $I_{bag}$ and $I_{ext}$ the first, resp. second, resp. third, resp. fourth component of a valuation $I$.
Let $\cI$ be the set of valuations over variables in $V_{int}\cup V_{\textit{uf}} \cup V_{ext} \cup V_{bag}$.
The semantics of QFBILIAUF is defined by
\begin{itemize}
\item a function $\ ^*: \cI\flc \cT_{int}\flc \ZZ$ mapping a valuation and an integer term to an integer value,
\item a function $\ ^\square: \cI\flc \cT_{\textit{uf}}\flc \ZZ $ mapping a valuation and an uninterpreted function term to an integer value,
\item a function $\ ^\circ: \cI\flc \cT_{bag}\flc \MM[\ZZ]$ mapping a valuation and a multi-set term to a multi-set value,
\item a function $\ ^{\bullet}: \cI\flc \cT_{ext}\flc \ZZ\cup\{\bot,\top\}$ mapping a valuation and an integer term to an extremum value,
\item a relation $\ \models \subseteq \cI\times \cF$ between valuations and formulas.
\end{itemize}
$I^{*}$, $I^{\bullet}$ and $I^{\circ}$ have the same definition as in \QFBILIA, see section \ref{ssec:sem}.
\\
Given a valuation $I$ and an uninterpreted function term $T_{\textit{uf}}$, the valuation of $T_{\textit{uf}}$ in $\ZZ$,
denoted by $I^{\square}(T_{\textit{uf}})$, is defined as follows:
\begin{align*}
I^{\square}(G(T_{int})) \triangleq&\ I_{\textit{uf}}(G)(I^*(T_{int})) \\
\end{align*}

\begin{mydef}
  A QFBILIAUF formula $F$ is \emph{satisfiable} if there exists a valuation $I$, called also its model, such that $I \models F$.
\end{mydef}

\subsection{Proof of Proposition~\ref{prop:order:countabs}}


%The following property states that $\tilde{F}^3 \sim_{sat} \tilde{F}^2$:
%\begin{myprop}
%\label{prop:order:countabs}
%(i) any model of $\tilde{F}^3$ is a model of $\tilde{F}^2$ and
%(ii) for any model of $\tilde{F}^2$, there is a model of $\tilde{F}^3$ that agrees on $V_{int}(\tilde{F}^2)$ and $V_{bag}(\tilde{F}^2)$.
%\end{myprop}

\begin{proof}

Because $\tilde{F}^3 \triangleq \tilde{F}^2 \land F_{3}$, any model $I$ of $\tilde{F}^3$ is also a model of $\tilde{F}^2$.

We prove now the point (ii):
\begin{enumerate}
\item[A.1:] $I$ be a model of $\tilde{F}^2$
\item[C.1:] There is $I'$ which agrees on $V(\tilde{F}^2)$ to $I$ and $I'$ is a model of $\tilde{F}^3$.
\end{enumerate}
Proof:
\begin{enumerate}
\item[1:] Let $I'=(I_{bag}, I_{int},I'_{\textit{uf}})$ such that for any $G_{x}\in V_{\textit{uf}}(\tilde{F}^3)$ and for any $k \in \ZZ$ we have
$$
I'_{\textit{uf}}(G_{x})(k) \triangleq \left\{
\begin{array}{ll}
I_{bag}(x)(a) & \mbox{ if }  I_{int}(a) = k \mbox{ for some } a \in V_{int}(\tilde{F}^2) \\
I_{bag}(x)(m) & \mbox{ if }  I_{ext}(m) = k \mbox{ for some } m \in V_{ext}(\tilde{F}^2) \\
I_{bag}(x)(v) & \mbox{ if }  I_{int}(a_{x\neq y}) = k \mbox{ for some }a_{x\neq y} \in V^{31}_{int} \\
& \quad \mbox{ and } I_{bag}(x)(v) \neq I_{bag}(y)(v)\\
I_{bag}(x)(v) & \mbox{ if }  I_{int}(a_{x\nsubseteq y}) = k \mbox{ for some }a_{x\nsubseteq y} \in V^{31}_{int} \\
& \quad \mbox{ and } I_{bag}(x)(v) > I_{bag}(y)(v)\\
0 & \mbox{ otherwise}\\
%I_{bag}(x)(b) & \mbox{ if } a \equiv w_{b,x} \in V^{32}_{int}
\end{array}\right.
$$
for any $m\in V_{ext}(\tilde{F}^3)$ we have
$$
I'_{ext}(m) \triangleq \left\{
\begin{array}{ll}
I_{ext}(m) & \mbox{ if } m \in V_{ext}(\tilde{F}^2) \\
k & \mbox{ if } m_{\bot} \mbox{ and } \forall a \in V_{int}(\tilde{F}^3), k<I'_{int}(a) \\
& \quad\quad\quad\mbox{ and } \forall m \in V_{ext}(\tilde{F}^3), k\leq I'_{ext}(m)\\
k & \mbox{ if } m_{\top} \mbox{ and } \forall a \in V_{int}(\tilde{F}^3), k>I'_{int}(a) \\
& \quad\quad\quad\mbox{ and } \forall m \in V_{ext}(\tilde{F}^3), k\geq I'_{ext}(m)\\
\end{array}\right.
$$

This valuation of integer and extremum variables satisfies the part $\tilde{F}^2$ of $\tilde{F}^3$ by (A.1)
and it satisfies the $F_{3}$ part by the properties of $I_{bag}$.
Thus, C.1. is satisfied.
\end{enumerate}
\end{proof}



\subsection{Proof of Proposition~\ref{prop:order:final}}

\input fragment-order-s4.tex

\begin{proof}
Clearly $\tilde{F}^4$ is in QFLIA from the definition of $S_{u}4$.

Point (i): Because $S_{u}4$ does not rewrite integer atoms and the $F_3$ conjunct, a model $I$ of $\tilde{F}^3$ satisfies, with its integer valuation $I_{int}$, the integer atoms not rewritten by $S_{u}4$.
The uninterpreted function atoms introduced by $S_{u}4$ are conform with the semantics of formulas while interpreting the variables $X(*)$ as counting abstractions of a bag $x$.
If $x \neq y$ is satisfied then there exists a value $v$ in $\ZZ$ such as $I_{bag}(x)(v) \neq I_{bag}(y)(v)$. By the choice of $I_{int}$ in the 3rd step (i.e., it satisfies $F_3$), we have that $I_{int}(a_{x\neq y}) = v$
If $m = max(x)$ is satisfied then either $I_{bag}(x) = \bagone{} $ and $I_{ext}(m) = I_{ext}(m_{\bot})$, or $I_{ext}(m)$ is the greatest integer of $\mathcal{D}(I_{bag}(x))$ but different from $I_{ext}(m_{\bot})$ and $I_{ext}(m_{\top})$.

Point (ii):
\begin{enumerate}
\item[A.1] $I$ is a model of $S_{u}4(\tilde{F}^3)$
\item[C.1] there is a model $I'$ of $\tilde{F}^3$ that agrees on $V_{int}(\tilde{F}^3)$
\end{enumerate}

Proof:
\begin{enumerate}[1.]
\item %[1.]
$I_{int}$ satisfies the $F_3$ component of $\tilde{F}^3$ which is preserved as it is by $S_{u}4$.
By the definition of $S_{u}4$, $V_{int}(\tilde{F}^3) = V_{int}(S_{u}4(\tilde{F}^3))$.

\item %[2.]
We define a valuation of bag variables in $\tilde{F}^3$, $I'_{bag}$ as follows:
%Same issue V_{int}(\tilde{F}^3)
$$
I'_{bag}(x)(v) \triangleq \left\{\begin{array}{ll}
I_{\textit{uf}}(G_{x})(a) & \mbox{ if } v = I_{int}(a) \mbox{ for some }a\in V^3_{int} \\
I_{\textit{uf}}(G_{x})(m) & \mbox{ if } v = I_{ext}(m) \mbox{ for some }m\in V^3_{ext} \\
0 & \mbox{ otherwise}
\end{array}\right.
$$
Then $I'_{bag}$ is a well funded function.

Proof: By the fact that $I_{\textit{uf}}$ satisfies the constraint $F_3$, which means that values of $I_{\textit{uf}}(G_{x})(*) \ge 0$.

\item %[3.]
$I'=(I_{int}(\tilde{F}^3),I'_{bag})$ is a model of $S_{u}4(\tilde{F}^3)$.

Proof: The proof proceeds by induction on the form of atoms transformed by $S_{u}4$. We show only the main points of the induction, the others are done similarly.

\begin{enumerate}[3.1]
\item %[3.1:]
for $L_{bag} ::=\ x = \llbracket a \rrbracket$, if $I \models S_{u}4(L_{bag})$ then $I'\models L_{bag}$ (and vice versa). \\

\begin{enumerate}[3.1.1]
\item %[3.1.1:]
From definition of $S_{u}4(L_{bag})$,
\begin{align*}
(I_{\textit{uf}}(G_{x})(I_{int}(a)) = 1)  
&\land \underset{b\in V^3_{int}}{\bigwedge}((I_{int}(a) = I_{int}(b)) \lor (I_{\textit{uf}}(G_{x})(I_{int}(b)) = 0)) \\
&\land \underset{m\in V^3_{ext}}{\bigwedge}((I_{int}(a) = I_{ext}(m)) \lor (I_{\textit{uf}}(G_{x})(I_{ext}(m)) = 0)).
\end{align*}

\item %[3.1.2:]
From 3.1.1 and the definition of $I'_{bag}$,
\begin{align*}
(I_{bag}(x)(I_{int}(a)) = 1)  
&\land \underset{b\in V^3_{int}}{\bigwedge}((I_{int}(a) = I_{int}(b)) \lor (I_{bag}(x)(I_{int}(b)) = 0)) \\
&\land \underset{m\in V^3_{ext}}{\bigwedge}((I_{int}(a) = I_{ext}(m)) \lor (I_{bag}(x)(I_{ext}(m)) = 0)).
\end{align*}

\item %[3.1.3:]
From 3.1.2 and the definition of $I'_{bag}$,
$$(I_{bag}(x)(I_{int}(a)) = 1) \land \underset {k\in \ZZ}{\bigwedge}((I_{int}(a) = k) \lor (I_{bag}(x)(k) = 0)).$$

\item %[3.1.4:]
From 3.1.3 the semantic 2.2 of $T_{bag}$ and $L_{bag}$,
$$I'_{bag}(x) = \bagone{I_{int}(a)}$$.

\end{enumerate}
%usefull ? we admit in first step ?
The reverse direction is shown similarly.

Hence, 3.1 is valid.


\item %[3.2:]
for $L_{bag}::=\ x \neq y$, if $I \models S_{u}4(L_{bag})$ then $I'\models L_{bag}$ (and vice versa). \\

\begin{enumerate}[3.2.1]
\item %[3.1.1:]
From definition of $S_{u}4(L_{bag})$,
$$I_{\textit{uf}}(G_{x})(I_{int}(a_{x\neq y})) \neq I_{\textit{uf}}(G_{y})(I_{int}(a_{x\neq y}))$$.

\item %[3.1.2:]
From 3.2.1 and the definition of $I'_{bag}$,
$$I_{bag}(x)(I_{int}(a_{x\neq y})) \neq I_{bag}(y)(I_{int}(a_{x\neq y}))$$.

\item %[3.1.3:]
From 3.2.2 and the definition of $I'_{bag}$,
$$\exists k \in \ZZ I_{bag}(x)(k) \neq I_{bag}(y)(k)$$.

\item %[3.1.4:]
From 3.2.3 and the semantic 2.2 of $L_{bag}$,
$$I'_{bag}(x) \neq I'_{bag}(y)$$.

\end{enumerate}
%usefull ? we admit in first step ?
The reverse direction is shown similarly.

Hence, 3.2 is valid.

\end{enumerate} %% end of  3

\item %[4.]
$I'=(I_{int},I'_{bag})\models S_{u}4(\tilde{F}^3)$

Proof: by 1, 2, and 3 above.


\item %[5.]
C.1

Proof: by 4 and the definition of $I'$.

\end{enumerate}
\end{proof}
