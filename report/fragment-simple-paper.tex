%!TEX root = paper.tex

\section{Decision Procedure by Reduction to QFLIA}
\label{sec:fsimple}

This decision procedure checks the satisfiability of a \QFBILIA\ formula $F$ in four steps.
These steps are removing gradually the multi-set atoms to obtain an equi-satisfiable formula in quantifier free linear arithmetics, i.e., the QFLIA theory of SMTlib.

The first step translates $F$ into a $\QFBILIA_\nnf$ formula $\tilde{F}$ which is equivalent with $F$, from  Proposition~\ref{prop:nnf}.
The second step transforms $\tilde{F}$ into an semi-equivalent formula $\tilde{F}^2$ in $\QFBILIA_{pure}$, from  Proposition~\ref{prop:pure}.
The third step introduces two extremum variables to represent $\bot$ and $\top$, two sets of fresh integer variables to represent
(i) the elements that validates atoms $x\neq y$ and $x\nsubseteq y$ and
(ii) for each multi-set variable $x\in V_{bag}(\tilde{F}^2)$ and integer variable $a\in V_{int}(\tilde{F}^2)$, resp. extremum variable $m\in V_{ext}(\tilde{F}^2)$,
 an integer variable $w_{a,x}$, resp. $w_{m,x}$ counting the occurrences of values of $a$, resp. $m$ in $x$.
This step only adds to $\tilde{F}^2$ a formula $F_3$ over integer variables. The formula $\tilde{F}^2 \land F_3$ is then equi-satisfiable to $F$.
The fourth step removes multi-set terms from $\tilde{F}^3$ using the fresh integer variables introduced by the counting abstraction and produces a formula $\tilde{F}^4 \land F_3$ equi-satisfaisable to $F$.



\subsection{First Step: Translation to $\QFBILIA_\nnf$}
\label{ssec:dp-s1}

The correctness of this step is stated by the following proposition whose proof gives a procedure to build a formula $\tilde{F}$ in $\QFBILIA_\nnf$ from a formula $F$ in \QFBILIA. The proof in provided in Appendix~\ref{proof:prop:nnf}.

\begin{myprop}
\label{prop:nnf}
For any $F$ formula in \QFBILIA, there exists a formula $\tilde{F}$ in $\QFBILIA_\nnf$ over a set of variables
$V_{int}(F)\cup V^\nnf_{int}$ and $V_{bag}(F)\cup V^\nnf_{bag}$ such that
(i) any model of $\tilde{F}$ is a model of $F$ and
(ii) for any model of $F$, there is a model of $\tilde{F}$ that agrees on $V_{int}(F)$ and $V_{bag}(F)$.
\end{myprop}

% \begin{comp}
% The $\nnf$ function is clearly linear among atom of the formula $\tilde{F}$ as we only deal with them once
% $\sigma(\nnf) = \sigma(|L(\tilde{F})|)$.
% The rewritting of $\ite(F',T^1, T^2)$ have to be avoid as much as possible, indeed let's denote $F_{b}$ the formula with $\ite(F',T^1, T^2)$ and $F_{a}$ the equivalent formula without $\ite(F',T^1, T^2)$
% then $|L(F_{a})| = |L(F_{b})| + 6 + |L(F')|$.
% see section \ref{sec:implementation}.
% \td{changer sigma}
% \end{comp}



\subsection{Second Step: Translation to $\QFBILIA_{pure}$}
\label{ssec:dp-s2}

The step applies the following sequence of transformations on $\tilde{F}$ in $\QFBILIA_\nnf$:
\begin{enumerate}
\item[S2.1:] Every bag atom $T^1 \prec T^2$ constraint (with $\prec \in \{<,\ge\}$) is rewritten using the $\min$ and $\max$ extremum terms as follows:
\begin{itemize}
\item $T^1 \geq T^2$ becomes $\min(T^1) \geq \max(T^2)$
\item $T^1 <    T^2$ becomes $\max(T^1) < \min(T^2)$
\end{itemize}
Thus, the bag atom becomes an extremum atom.
Let denote by $\tilde{F}_{2.1}$ the result of this process.

\item[S2.2:] Every $T_{bag}$ term used in \emph{integer, mixed, extremum} atoms which is not a variable is replaced by a fresh multi-set variable $x$ and the multi-set atom $x=T_{bag}$ is conjuncted to $\tilde{F}_{2.1}$.
Let us denote by $\tilde{F}_{2.2}$ the result of this process and by $V^{2.2}_{bag}$ the set of fresh bag variables.

\item[S2.3:] Every extremum term of the form $\min(x)$ or $\max(x)$ is replaced by a fresh extremum variable $\min_x$ rep. $\max_x$ and the mixed atom $\min(x) = \min_x$ is conjuncted to $\tilde{F}_{2.2}$.
Let us denote by $\tilde{F}_{2.3}$ the result of this process and by $V^{2.3}_{ext}$ the set of fresh extremum variables.

\item[S2.4:] Every bag atom $L_{bag}$ of $\tilde{F}_{2.3}$ using more than two bag variables for operations $\neq, \subseteq, \nsubseteq$ and more than three variables for operation $=$ are iteratively rewritten to be reduced to the bag atoms in $\QFBILIA_{pure}$. Notice that integer, mixed and extremum atoms already use only bag variables and not bag terms due to the step S2.2.
So, at the end of this step, the resulting formula $\tilde{F}_{2.4}$ is in the $\QFBILIA_{pure}$ fragment. The set of fresh bag variables is denoted by $V^{2.4}_{bag}$
\end{enumerate}
The formula $\tilde{F}_{2.4}$ has the set of integer variables $V_{int}(\tilde{F})$,
the set of extremum variables $V_{ext}(\tilde{F}) \cup V^{2.3}_{ext}$ and
the set of bag variables $V_{bag}(\tilde{F}) \cup V^{2.2}_{bag} \cup V^{2.4}_{bag}$.

\begin{myex}
%\td{Explain why this example is given here.}
We explain the previously mentioned steps on the following $\QFBILIA_\nnf$ formula:
$$\min(x \cup y \cup z) = \min(x \cap y \cap z).$$ 
It is translated in the following $\QFBILIA_{pure}$ formula over the set of bag variables $\{x,y,z\}\cup\{xunionyunionz,yunionz,xinteryinterz,yinterz\}$
and the set of integer variables $\{min_{xunionyunionz},min_{xinteryinterz}\}$ :
\begin{align*}
min_{xunionyunionz} = min_{xinteryinterz} \ \land \\
  min_{xunionyunionz} = min(xunionyunionz) \  \land \\
  min_{xinteryinterz} = min(xinteryinterz) \ \land \\
  xunionyunionz = x \cup yunionz \ \land
  yunionz = y \cup z \ \land \\
  xinteryinterz = x \cap yinterz \ \land
  yinterz = y \cap z
\end{align*}
\end{myex}

The correctness of this transformation with respect to the semi-equivalence  is stated by the following result, which proof is given in the appendix.

\begin{myprop}
\label{prop:pure}
For any $\tilde{F}$ formula in $\QFBILIA_\nnf$, there exists a formula $\tilde{F}^2$ in $\QFBILIA_{pure}$ over a set of variables
$V_{int}(\tilde{F})$, $V_{ext}(\tilde{F})\cup V^p_{ext}$ and $V_{bag}(\tilde{F})\cup V^p_{bag}$ such that
(i) any model of $\tilde{F}^2$ is a model of $\tilde{F}$ and
(ii) for any model of $\tilde{F}$, there is a model of $\tilde{F}^2$ that agrees on $V_{int}(\tilde{F})$ and $V_{bag}(\tilde{F})$.
\end{myprop}


\subsection{Third Step: Introducing the Counting Abstraction}

This step adds to $\tilde{F}^2$ a formula $F_3$ that introduces the integer variables which will replace the multi-set variables. The transformation has three steps:

\begin{enumerate}
\item[S3.1:] Build the set $V^{31}_{int}$ as a set of fresh variables, one variable for each atom $(x \neq y)$ or $(x \nsubseteq y)$ in $\tilde{F}^2$.
We denote these variables by $a_{x \neq y}$, resp. $a_{x \nsubseteq y}$.
Intuitively, these variables are introduced to be able to express the fact that there is a value on which $x$ and $y$ differ, resp. $x$ has more copies than $y$ (see subsection \ref{subsec:rewritting qflia}).

\item[S3.2:] We introduce two fresh extremum variables $m_{\bot}$ and $m_{\top}$.
Intuitively, these variables are introduced to be able to express $\bot$, resp. $\top$.

\item[S3.3:] Build the set
$$V^{33}_{int}  =
\bigcup_{x\in V_{bag}(\tilde{F}^2), a\in  V^{31}_{int}\cup V_{int}(\tilde{F}^2)} w_{a,x},$$
where an integer variable is added for each pair of bag variable $x$ and integer variable $a$ in order to represent the number of element $a$ in $x$.
$$V^{33}_{ext}  =
\bigcup_{x\in V_{bag}(\tilde{F}^2), m\in  V_{ext}(\tilde{F}^2)} w_{m,x},$$
where an integer variable is added for each pair of bag variable $x$ and extremum variable $m$ in order to represent the number of element $m$ in $x$.
Let $V^3_{int} = V_{int}(\tilde{F}^2) \cup V^{31}_{int}$.
Let $V^3_{ele} = V^3_{int} \cup V_{ext}(\tilde{F}^2)$.

The formula $\tilde{F}^3$ is built as follows:
\begin{small}\begin{align}
  \tilde{F}^3 \triangleq &\
  \tilde{F}^2 \land F_{3} \\
  F_{3 } \triangleq &\
  \left(\bigwedge_{a \in V^{3}_{int}}(m_{\bot} < a < m_{\top})\right)
  \\
  & \land
  \left(\bigwedge_{m \in V_{ext}(\tilde{F}^2)}(m_{\top} \ge m \ge m_{\bot})\right)
  \\
  & \land
  %\bigwedge_{x \in V_{bag}(\tilde{F}^2)}\left(\land_{a\in V^3_{int}}w_{a,x} \ge 0\right)  WRONG DU TO w_{w_a,x,x}
  \left(\bigwedge_{w \in V^{33}_{int}\cup V^{33}_{ext}}(w \ge 0)\right)
  \\
  & \land
  \left(\bigwedge_{t,u \in V^3_{ele}}\left(\bigwedge_{x\in V_{bag}(\tilde{F}^2)}(t \neq u) \lor (w_{t,x} = w_{u,x})\right)\right)
\end{align}
\end{small}
The set of variables of $\tilde{F}^3$ are $V_{int}(\tilde{F}^3)$, resp. $V_{ext}(\tilde{F}^3)$, resp. $V_{bag}(\tilde{F}^2)$.
\end{enumerate}

The following property states that $\tilde{F}^3 \sim_{sat} \tilde{F}^2$:
\begin{myprop}
\label{prop:countabs}
The formula $\tilde{F}^3$ is in $\QFBILIA_{pure}$ and
(i) any model of $\tilde{F}^3$ is a model of $\tilde{F}^2$ and
(ii) for any model of $\tilde{F}^2$, there is a model of $\tilde{F}^3$ that agrees on $V_{int}(\tilde{F}^2)$, $V_{ext}(\tilde{F}^2)$ and $V_{bag}(\tilde{F}^2)$.
\end{myprop}

\subsection{Fourth Step: Removing Multi-set Constraints}
\label{subsec:rewritting qflia}
This step rewrites the bag atoms and the mixed atoms in order to remove the bag terms and so transforms $\tilde{F}^3$ to a formula in QFLIA.
The transformation preserves the boolean structure of the initial formulas and the integer atoms.
It is given by the $S4$ function defined in Figure~\ref{fig:S4}.

\begin{figure}
\begin{mdframed}
  \fontsize{9}{6}
%\begin{align*}
% f1,f2 \in F,S4(F_1 \land F_2) \triangleq &\ \   S4(f1) \land S4(f2) \\
% f1,f2 \in F,S4(F_1 \lor F_2)  \triangleq &\ \   S4(f1) \lor S4(f2)
%\end{align*}
%
Translation of mixed atoms:
%J'ai un soucis le V_{int}(\tilde{F}^3)}{\bigwedge} car c'est faut il va itéré sur les w aussi :s
%Mais V^3_{int} n'existe pas :s ou je sais pas :s
\begin{align*}
 S4(m = max(x))  \triangleq &\ \   \left((m = m_{\bot}) \land \underset{t\in V^3_{ele}}{\bigwedge}(w_{t,x} = 0)\right) \lor
 \\ & \ \  \left((m \neq m_{\bot}) \land (m \neq m_{\top}) \land (w_{m,x} \ge 1) \land \underset{t\in V^3_{ele}}{\bigwedge}((t \le m)  \lor (w_{t,x} = 0))\right)
\\
  S4(m = min(x))  \triangleq &\ \   \left((m = m_{\top}) \land \underset{t\in V^3_{ele}}{\bigwedge}(w_{t,x} = 0)\right) \lor
  \\ & \ \  \left((m \neq m_{\bot}) \land (m \neq m_{\top}) \land (w_{m,x} \ge 1) \land \underset{t\in V^3_{ele}}{\bigwedge}((t \ge m)  \lor (w_{t,x} = 0))\right)
\\
 S4(a \in x )    \triangleq &\ \   w_{a,x} \ge 1 \\
 S4(a \in^{n} x) \triangleq &\ \   w_{a,x} \ge n \\
 S4(a \notin x ) \triangleq &\ \   w_{a,x} = 0 \\
 S4(a \notin^{n} x )  \triangleq &\ \   w_{a,x} < n
 \\
  S4(m \in x )    \triangleq &\ \   (m = m_{\bot} \lor m = m_{\top}) \land \underset{t\in V^3_{ele}}{\bigwedge}(w_{t,x} = 0) \lor w_{m,x} \ge 1 \\
  S4(m \in^{n} x) \triangleq &\ \   m \neq m_{\bot} \land m \neq m_{\top} \land w_{m,x} \ge n \\
  S4(m \notin x ) \triangleq &\ \   (m \neq m_{\bot} \land m \neq m_{\top}) \lor \underset{t\in V^3_{ele}}{\bigvee}(w_{t,x} \neq 0) \land w_{m,x} = 0 \\
  S4(m \notin^{n} x )  \triangleq &\ \   m = m_{\bot} \lor m = m_{\top} \lor w_{m,x} < n
\end{align*}

Translation of bag atoms:

\begin{align*}
 S4(x = y)           \triangleq &\ \   \underset{t \in V^3_{ele}}{\bigwedge}(w_{t,x} = w_{t,y})   \\
 S4(x \subseteq y)   \triangleq &\ \   \underset{t \in V^3_{ele}}{\bigwedge}(w_{t,x} \le w_{t,y})   \\
 S4(x \neq y)        \triangleq &\ \   w_{a_{x \neq y},x} \neq w_{a_{x \neq y},y}   \\
 S4(x \nsubseteq y)  \triangleq &\ \   w_{a_{x \nsubseteq y},x} > w_{a_{x \nsubseteq y},y}   \\
 S4(x = \bagone{} )  \triangleq &\ \   \underset{t\in V^3_{ele}}{\bigwedge}(w_{t,x} = 0)  \\
 S4(x = \bagone{a})  \triangleq &\ \   (w_{a,x} = 1)  \land \underset{t\in V^3_{ele}}{\bigwedge}((a = t) \lor (w_{t,x} = 0))    \\
 S4(x = y \cup z)    \triangleq &\ \   \underset{t\in V^3_{ele}}{\bigwedge}(w_{t,x} = max(w_{t,y},w_{t,z}))   \\
 S4(x = y \cap z)    \triangleq &\ \   \underset{t\in V^3_{ele}}{\bigwedge}(w_{t,x} = min(w_{t,y},w_{t,z}))   \\
 S4(x = y \uplus z)   \triangleq &\ \   \underset{t\in V^3_{ele}}{\bigwedge}(w_{t,x} = w_{t,y}+w_{t,z})   \\
 S4(x = y \setminus z) \triangleq &\ \   \underset{t\in V^3_{ele}}{\bigwedge}(w_{t,x} = \max(0,(w_{t,y}-w_{t,z})))
\end{align*}
\caption{Translation of mixed and bags atoms}
\label{fig:S4}
\end{mdframed}
\end{figure}

\begin{myprop}
\label{prop:final}
The formula $\tilde{F}^4 = S4(\tilde{F}^3)$ is in QFLIA and
%De meme c'est vrai si w = 0 pour tout w par exemple. on doit imposer la creation des w
(i) for any model of $I$ of $\tilde{F}^3$ there is a model $I'$ of $\tilde{F}^4$ such that $I$ and $I'$ agree on $V^3(\int)$.
(ii) for any model $I$ of $\tilde{F}^4$, there is a model $I'$ of $\tilde{F}^3$ such that $I$ and $I'$ agree on $V_{int}(\tilde{F}^3)$.
\end{myprop}

