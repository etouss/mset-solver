%!TEX root = main.tex

\section{Decision procedure by reduction to QFLIA}
\label{sec:fsimple}

The decision procedure for checking satisfiability of a \QFBILIA\ formula $F$ proceeds in four steps.
These steps are removing gradually the multi-set atoms to obtain an equi-satisfiable formula in quantifier free linear arithmetics (QFLIA theory).

The first step translates $F$ into a $\QFBILIA_\nnf$ formula $\tilde{F}$ which is equivalent with $F$, from  Proposition~\ref{prop:nnf}.
The second step transforms $\tilde{F}$ into an semi-equivalent formula $\tilde{F}^2$ in $\QFBILIA_{pure}$, from  Proposition~\ref{prop:pure}.
The third step introduces two extremum variables to represent $\bot$ and $\top$, two sets of fresh integer variables to represent
(i) the elements that validates atoms $x\neq y$ and $x\nsubseteq y$ and
(ii) for each multi-set variable $x\in V_{bag}(\tilde{F}^2)$ and integer variable $a\in V_{int}(\tilde{F}^2)$, resp. extremum variable $m\in V_{ext}(\tilde{F}^2)$,
 an integer variable $w_{a,x}$, resp. $w_{m,x}$ counting the occurrences of values of $a$, resp. $m$ in $x$.
This step only adds to $\tilde{F}^2$ a formula $F_3$ over integer variables. The formula $\tilde{F}^2 \land F_3$ is then equi-satisfiable to $F$.
The fourth step removes multi-set terms from $\tilde{F}^3$ using the fresh integer variables introduced by the counting abstraction and produces a formula $\tilde{F}^4 \land F_3$ equi-satisfaisable to $F$.



\subsection{First Step: Translation to $\QFBILIA_\nnf$}
\label{ssec:dp-s1}

The correctness of this step is stated by the following proposition whose proof gives a procedure to build a formula $\tilde{F}$ in $\QFBILIA_\nnf$ from a formula $F$ in \QFBILIA.

\begin{myprop}
\label{prop:nnf}
For any $F$ formula in \QFBILIA, there exists a formula $\tilde{F}$ in $\QFBILIA_\nnf$ over a set of variables
$V_{int}(F)\cup V^\nnf_{int}$ and $V_{bag}(F)\cup V^\nnf_{bag}$ such that
(i) any model of $\tilde{F}$ is a model of $F$ and
(ii) for any model of $F$, there is a model of $\tilde{F}$ that agrees on $V_{int}(F)$ and $V_{bag}(F)$.
\end{myprop}

\begin{proof}
The proof is given by construction of $\tilde{F}$ from $F$.

% First every term $\ite(F',T^1, T^2)$ is replaced by the formula
% \\$(F' \cup T^2) \land (\neg F' \cup T^1)$
First, every integer term $\ite(F',T^1, T^2)$ is replaced by a fresh integer variable $a$ (i.e., not in $V_{int}(F)$)
and $F$ is rewritten in $F \land (F' \limp a=T^1) \land \big((\lnot F') \limp (a=T^2)\big)$.
The same procedure is applied for $\ite$ terms in multi-set atoms.
Using the semantics of $\ite$ terms, we show that the formula $\tilde{F}$ obtained by this rewriting satisfies the conclusion of the theorem.

Second, we apply to $\tilde{F}$ the de Morgan's rules to eliminate the negation by pushing them at the level of literals.
We definite inductively on the formula syntax a function nnf of $\QFBILIA \flc \QFBILIA_\nnf$,
for any $F_1,F_2 \in \QFBILIA$ and $\ell\in L$, as follows:
\begin{flalign}
  \nnf(F_1 \limp F_2)  &=    \nnf(\lnot F_1) \lor \nnf(F_2)    \\
  \nnf(\lnot(F_1 \limp F_2))  &=    \nnf(F_1) \land \nnf(\lnot F_2)    \\
  \nnf(F_1 \lor F_2)    &= \nnf(F_1) \lor \nnf(F_2)   \\
  \nnf(\lnot (F_1 \lor F_2)  &=   \nnf(\lnot F_1) \land \nnf(\lnot F_2)   \\
  \nnf(F_1 \land F_2)  &=   \nnf(F_1) \land \nnf(F_2)   \\
  \nnf(\lnot (F_1 \land F_2)  &=    \nnf(\lnot F_1) \lor \nnf(\lnot F_2)   \\
  \nnf(\lnot\lnot F)  &= \nnf(F)   \\
  \nnf(\ell)  &= \ell   \\
  \nnf(\lnot \ell)  &=    \tilde{\ell} \mbox{ i.e., the opposite atom of }\ell   \\
\end{flalign}
Since any atom has its opposice in $\QFBILIA_\nnf$, and since $(F\land F)$, $(F \lor F)$ are also formulas in $\QFBILIA_\nnf$, then
$\nnf(F) \in \QFBILIA_\nnf$, (for instace $\lnot (T_{int}^1 < T_{int}^2)$ is replaced by $(T_{int}^1 \geq T_{int}^2)$).
\end{proof}

% \begin{comp}
% The $\nnf$ function is clearly linear among atom of the formula $\tilde{F}$ as we only deal with them once
% $\sigma(\nnf) = \sigma(|L(\tilde{F})|)$.
% The rewritting of $\ite(F',T^1, T^2)$ have to be avoid as much as possible, indeed let's denote $F_{b}$ the formula with $\ite(F',T^1, T^2)$ and $F_{a}$ the equivalent formula without $\ite(F',T^1, T^2)$
% then $|L(F_{a})| = |L(F_{b})| + 6 + |L(F')|$.
% see section \ref{sec:implementation}.
% \td{changer sigma}
% \end{comp}



\subsection{Second Step: Translation to $\QFBILIA_{pure}$}
\label{ssec:dp-s2}

The step applies the following sequence of transformations on $\tilde{F}$ in $\QFBILIA_\nnf$:
\begin{enumerate}
\item[S2.1:] Every bag atom $T^1 \prec T^2$ constraint (with $\prec \in \{<,\ge\}$) is rewritten using the $\min$ and $\max$ extremum terms as follows:
\begin{itemize}
\item $T^1 \geq T^2$ becomes $\min(T^1) \geq \max(T^2)$
\item $T^1 <    T^2$ becomes $\max(T^1) < \min(T^2)$
\end{itemize}
Thus, the bag atom becomes an extremum atom.
Let denote by $\tilde{F}_{2.1}$ the result of this process.

\item[S2.2:] Every $T_{bag}$ term used in \emph{integer, mixed, extremum} atoms which is not a variable is replaced by a fresh multi-set variable $x$ and the multi-set atom $x=T_{bag}$ is conjuncted to $\tilde{F}_{2.1}$.
Let us denote by $\tilde{F}_{2.2}$ the result of this process and by $V^{2.2}_{bag}$ the set of fresh bag variables.

\item[S2.3:] Every extremum term of the form $\min(x)$ or $\max(x)$ is replaced by a fresh extremum variable $\min_x$ rep. $\max_x$ and the mixed atom $\min(x) = \min_x$ is conjuncted to $\tilde{F}_{2.2}$.
Let us denote by $\tilde{F}_{2.3}$ the result of this process and by $V^{2.3}_{ext}$ the set of fresh extremum variables.

\item[S2.4:] Every bag atom $L_{bag}$ of $\tilde{F}_{2.3}$ using more than two bag variables for operations $\neq, \subseteq, \nsubseteq$ and more than three variables for operation $=$ are iteratively rewritten to be reduced to the bag atoms in $\QFBILIA_{pure}$. Notice that integer, mixed and extremum atoms already use only bag variables and not bag terms due to the step S2.2.
So, at the end of this step, the resulting formula $\tilde{F}_{2.4}$ is in the $\QFBILIA_{pure}$ fragment. The set of fresh bag variables is denoted by $V^{2.4}_{bag}$
\end{enumerate}
The formula $\tilde{F}_{2.4}$ has the set of integer variables $V_{int}(\tilde{F})$,
the set of extremum variables $V_{ext}(\tilde{F}) \cup V^{2.3}_{ext}$ and
the set of bag variables $V_{bag}(\tilde{F}) \cup V^{2.2}_{bag} \cup V^{2.4}_{bag}$.

\begin{myex}
%\td{Explain why this example is given here.}
We explain the previously mentioned steps on the following $\QFBILIA_\nnf$ formula:
$$\min(x \cup y \cup z) = \min(x \cap y \cap z).$$ 
It is translated in the following $\QFBILIA_{pure}$ formula over the set of bag variables $\{x,y,z\}\cup\{xunionyunionz,yunionz,xinteryinterz,yinterz\}$
and the set of integer variables $\{min_{xunionyunionz},min_{xinteryinterz}\}$ :
\begin{align*}
min_{xunionyunionz} = min_{xinteryinterz} \ \land \\
  min_{xunionyunionz} = min(xunionyunionz) \  \land \\
  min_{xinteryinterz} = min(xinteryinterz) \ \land \\
  xunionyunionz = x \cup yunionz \ \land
  yunionz = y \cup z \ \land \\
  xinteryinterz = x \cap yinterz \ \land
  yinterz = y \cap z
\end{align*}
\end{myex}

The correctness of this transformation with respect to the semi-equivalence  is stated by the following result.

\begin{myprop}
\label{prop:pure}
For any $\tilde{F}$ formula in $\QFBILIA_\nnf$, there exists a formula $\tilde{F}^2$ in $\QFBILIA_{pure}$ over a set of variables
$V_{int}(\tilde{F})$, $V_{ext}(\tilde{F})\cup V^p_{ext}$ and $V_{bag}(\tilde{F})\cup V^p_{bag}$ such that
(i) any model of $\tilde{F}^2$ is a model of $\tilde{F}$ and
(ii) for any model of $\tilde{F}$, there is a model of $\tilde{F}^2$ that agrees on $V_{int}(\tilde{F})$ and $V_{bag}(\tilde{F})$.
\end{myprop}

\begin{proof}
The proof is given by construction of $\tilde{F}^2$ from $\tilde{F}$ following the steps S2.1--4 above.

The formula $\tilde{F}_{2.1}$ is equivalent to $\tilde{F}$ due to the definition of the semantics of comparison operations over multi-sets.

The formula $\tilde{F}_{2.2}$ is built over the set of variables
$V_{int}(\tilde{F})$ and $V_{bag}(\tilde{F})\cup V^{2.2}_{bag}$
where $V^{2.2}_{bag}$ is the set of fresh variables used to replace multi-set terms which are not variables in integer, extremum or mixed atoms. The semantics of equality between bags ensures that the equi-satisfiability is preserved.

The formula $\tilde{F}_{2.3}$ is built over the set of variables
$V_{int}(\tilde{F}_{2.2})$, $V_{ext}(\tilde{F}_{2.2})\cup V^{2.3}_{ext}$ and $V_{bag}(\tilde{F}_{2.2})$
where $V^{2.3}_{ext}$ is the set of fresh variables used to replace extremum term of the form $\min(x)$ or $\max(x)$. The semantics of equality between integer ensures that the equi-satisfiability is preserved.

The formula $\tilde{F}_{2.4}$ is built over the set of variables
$V_{int}(\tilde{F}_{2.3})$ and $V_{bag}(\tilde{F}_{2.3})\cup V^{2.4}_{bag}$
where $V^{2.4}_{bag}$ is the set of fresh variables used to replace multi-set terms in bag atoms which don't fit in the $\QFBILIA_{pure}$ fragment. The semantics of equality between bags ensures that the equi-satisfiability is preserved, i.e., the two conclusions are valid.

The formula $\tilde{F}^2 = \tilde{F}_{2.4}$ is in $\QFBILIA_{pure}$ and the two conclusions are valid as a consequence of steps S2.1--4.

%\td{improve and continue the proof}
\end{proof}

% \begin{comp}
%   The procedure described in steps S2.1--4 can be done in a single treatement as consequence this step is linear among atom of $\tilde{F}$
%   $\sigma(S2) = \sigma(|L(\tilde{F})|)$.
%   However $|L(\tilde{F})| < |L(\tilde{F}^2)|$ indeed S2.2--4 generate variables and equalities are conjuncted.
%   see section \ref{sec:implementation}.
%   \td{sigma other}
% \end{comp}


\subsection{Third Step: Introducing the Counting Abstraction}

This step adds to $\tilde{F}^2$ a formula $F_3$ that introduces the integer variables which will replace the multi-set variables. The transformation has three steps:

\begin{enumerate}
\item[S3.1:] Build the set $V^{31}_{int}$ as a set of fresh variables, one variable for each atom $(x \neq y)$ or $(x \nsubseteq y)$ in $\tilde{F}^2$.
We denote these variables by $a_{x \neq y}$, resp. $a_{x \nsubseteq y}$.
Intuitively, these variables are introduced to be able to express the fact that there is a value on which $x$ and $y$ differ, resp. $x$ has more copies than $y$ (see subsection \ref{subsec:rewritting qflia}).

\item[S3.2:] We introduce two fresh extremum variables $m_{\bot}$ and $m_{\top}$.
Intuitively, these variables are introduced to be able to express $\bot$, resp. $\top$.

\item[S3.3:] Build the set
$$V^{33}_{int}  =
\bigcup_{x\in V_{bag}(\tilde{F}^2), a\in  V^{31}_{int}\cup V_{int}(\tilde{F}^2)} w_{a,x},$$
where an integer variable is added for each pair of bag variable $x$ and integer variable $a$ in order to represent the number of element $a$ in $x$.
$$V^{33}_{ext}  =
\bigcup_{x\in V_{bag}(\tilde{F}^2), m\in  V_{ext}(\tilde{F}^2)} w_{m,x},$$
where an integer variable is added for each pair of bag variable $x$ and extremum variable $m$ in order to represent the number of element $m$ in $x$.
Let $V^3_{int} = V_{int}(\tilde{F}^2) \cup V^{31}_{int}$.
Let $V^3_{ele} = V^3_{int} \cup V_{ext}(\tilde{F}^2)$.

The formula $\tilde{F}^3$ is built as follows:
\begin{align}
  \tilde{F}^3 \triangleq &\
  \tilde{F}^2 \land F_{3} \\
  F_{3 } \triangleq &\
  \left(\bigwedge_{a \in V^{3}_{int}}(m_{\bot} < a < m_{\top})\right)
  \\
  & \land
  \left(\bigwedge_{m \in V_{ext}(\tilde{F}^2)}(m_{\top} \ge m \ge m_{\bot})\right)
  \\
  & \land
  %\bigwedge_{x \in V_{bag}(\tilde{F}^2)}\left(\land_{a\in V^3_{int}}w_{a,x} \ge 0\right)  WRONG DU TO w_{w_a,x,x}
  \left(\bigwedge_{w \in V^{33}_{int}\cup V^{33}_{ext}}(w \ge 0)\right)
  \\
  & \land
  \left(\bigwedge_{t,u \in V^3_{ele}}\left(\bigwedge_{x\in V_{bag}(\tilde{F}^2)}(t \neq u) \lor (w_{t,x} = w_{u,x})\right)\right)
  \\
\end{align}
The set of variables of $\tilde{F}^3$ are $V_{int}(\tilde{F}^3)$, resp. $V_{ext}(\tilde{F}^3)$, resp. $V_{bag}(\tilde{F}^2)$.
\end{enumerate}

The following property states that $\tilde{F}^3 \sim_{sat} \tilde{F}^2$:
\begin{myprop}
\label{prop:countabs}
The formula $\tilde{F}^3$ is in $\QFBILIA_{pure}$ and
(i) any model of $\tilde{F}^3$ is a model of $\tilde{F}^2$ and
(ii) for any model of $\tilde{F}^2$, there is a model of $\tilde{F}^3$ that agrees on $V_{int}(\tilde{F}^2)$, $V_{ext}(\tilde{F}^2)$ and $V_{bag}(\tilde{F}^2)$.
\end{myprop}

\begin{proof}
Clearly $\tilde{F}^3$ is still in $\QFBILIA_{pure}$.

Moreover, because $\tilde{F}^3 \triangleq \tilde{F}^2 \land F_{3}$, any model $I$ of $\tilde{F}^3$ is also a model of $\tilde{F}^2$.

We prove now the point (ii):
\begin{enumerate}
\item[A.1:] $I$ be a model of $\tilde{F}^2$
\item[C.1:] There is $I'$ which agrees on $V(\tilde{F}^2)$ to $I$ and $I'$ is a model of $\tilde{F}^3$.
\end{enumerate}
Proof:
\begin{enumerate}
\item[1:] Let $I'=(I_{bag}, I'_{int}, I'_{ext})$ such that for any $a\in V_{int}(\tilde{F}^3)$ we have
$$
I'_{int}(a) \triangleq \left\{
\begin{array}{ll}
I_{int}(a) & \mbox{ if } a \in V_{int}(\tilde{F}^2) \\
v & \mbox{ if } a_{x\neq y} \in V^{31}_{int} \mbox{ and } I_{bag}(x)(v) \neq I_{bag}(y)(v) \\
v & \mbox{ if } a_{x\nsubseteq y} \in V^{31}_{int} \mbox{ and } I_{bag}(x)(v) > I_{bag}(y)(v) \\
I_{bag}(x)(b) & \mbox{ if } a \equiv w_{b,x} \in V^{32}_{int} \\
0 & \mbox{ otherwise} %Cas utile si I_{bag}(x)(v) \neq I_{bag}(y)(v) n'existe pas 0 n'importe quelle valeur fonctionne
\end{array}\right.
$$
for any $m\in V_{ext}(\tilde{F}^3)$ we have
$$
I'_{ext}(m) \triangleq \left\{
\begin{array}{ll}
I_{ext}(m) & \mbox{ if } m \in V_{ext}(\tilde{F}^2) \\
k & \mbox{ if } m_{\bot} \mbox{ and } \forall a \in V_{int}(\tilde{F}^3), k<I'_{int}(a) \\
& \quad\quad\quad \mbox{ and } \forall m \in V_{ext}(\tilde{F}^3), k\leq I'_{ext}(m)\\
k & \mbox{ if } m_{\top} \mbox{ and } \forall a \in V_{int}(\tilde{F}^3), k>I'_{int}(a) \\
& \quad\quad\quad \mbox{ and } \forall m \in V_{ext}(\tilde{F}^3), k\geq I'_{ext}(m)\\
\end{array}\right.
$$

This valuation of integer and extremum variables satisfies the part $\tilde{F}^2$ of $\tilde{F}^3$ by (A.1)
and it satisfies the $F_{3}$ part by the properties of $I_{bag}$.
Thus, C.1. is satisfied.
\end{enumerate}
\end{proof}

% \begin{comp}
% First of all the procedure described above adds a variable for each atom $(x \neq y)$ or $(x \nsubseteq y)$ in $\tilde{F}^2$.
% Then the procedure adds $|V_{bag}(\tilde{F}^2)| * |V^{31}_{int}\cup V_{int}(\tilde{F}^2)|$ variables.
% Finaly the procedure conjuncted $|V^{31}_{int}\cup V_{int}(\tilde{F}^2)|^{2} * |V_{bag}(\tilde{F}^2)|$ atoms to the formula.
% Then it is interesting to be abled to reduce $|V_{int}(\tilde{F}^2)|$, $|V^{31}_{int}|$ and $|V_{int}(\tilde{F}^2)|$ to a minimum (see section \ref{sec:implementation}).
% \end{comp}

\subsection{Fourth Step: Removing Multi-set Constraints}
\label{subsec:rewritting qflia}
This step rewrites the bag atoms and the mixed atoms in order to remove the bag terms and so transforms $\tilde{F}^3$ to a formula in QFLIA.
The transformation preserves the boolean structure of the initial formulas and the integer atoms.
It is given by the $S4$ function defined in Figure~\ref{fig:S4}.

\begin{figure}
\begin{mdframed}
  \fontsize{9}{6}
%\begin{align*}
% f1,f2 \in F,S4(F_1 \land F_2) \triangleq &\ \   S4(f1) \land S4(f2) \\
% f1,f2 \in F,S4(F_1 \lor F_2)  \triangleq &\ \   S4(f1) \lor S4(f2)
%\end{align*}
%
Translation of mixed atoms:
%J'ai un soucis le V_{int}(\tilde{F}^3)}{\bigwedge} car c'est faut il va itéré sur les w aussi :s
%Mais V^3_{int} n'existe pas :s ou je sais pas :s
\begin{align*}
 S4(m = max(x))  \triangleq &\ \   \left((m = m_{\bot}) \land \underset{t\in V^3_{ele}}{\bigwedge}(w_{t,x} = 0)\right) \lor
 \\ & \ \  \left((m \neq m_{\bot}) \land (m \neq m_{\top}) \land (w_{m,x} \ge 1) \land \underset{t\in V^3_{ele}}{\bigwedge}((t \le m)  \lor (w_{t,x} = 0))\right)
\\
  S4(m = min(x))  \triangleq &\ \   \left((m = m_{\top}) \land \underset{t\in V^3_{ele}}{\bigwedge}(w_{t,x} = 0)\right) \lor
  \\ & \ \  \left((m \neq m_{\bot}) \land (m \neq m_{\top}) \land (w_{m,x} \ge 1) \land \underset{t\in V^3_{ele}}{\bigwedge}((t \ge m)  \lor (w_{t,x} = 0))\right)
\\
 S4(a \in x )    \triangleq &\ \   w_{a,x} \ge 1 \\
 S4(a \in^{n} x) \triangleq &\ \   w_{a,x} \ge n \\
 S4(a \notin x ) \triangleq &\ \   w_{a,x} = 0 \\
 S4(a \notin^{n} x )  \triangleq &\ \   w_{a,x} < n
 \\
  S4(m \in x )    \triangleq &\ \   (m = m_{\bot} \lor m = m_{\top}) \land \underset{t\in V^3_{ele}}{\bigwedge}(w_{t,x} = 0) \lor w_{m,x} \ge 1 \\
  S4(m \in^{n} x) \triangleq &\ \   m \neq m_{\bot} \land m \neq m_{\top} \land w_{m,x} \ge n \\
  S4(m \notin x ) \triangleq &\ \   (m \neq m_{\bot} \land m \neq m_{\top}) \lor \underset{t\in V^3_{ele}}{\bigvee}(w_{t,x} \neq 0) \land w_{m,x} = 0 \\
  S4(m \notin^{n} x )  \triangleq &\ \   m = m_{\bot} \lor m = m_{\top} \lor w_{m,x} < n
\end{align*}

Translation of bag atoms:

\begin{align*}
 S4(x = y)           \triangleq &\ \   \underset{t \in V^3_{ele}}{\bigwedge}(w_{t,x} = w_{t,y})   \\
 S4(x \subseteq y)   \triangleq &\ \   \underset{t \in V^3_{ele}}{\bigwedge}(w_{t,x} \le w_{t,y})   \\
 S4(x \neq y)        \triangleq &\ \   w_{a_{x \neq y},x} \neq w_{a_{x \neq y},y}   \\
 S4(x \nsubseteq y)  \triangleq &\ \   w_{a_{x \nsubseteq y},x} > w_{a_{x \nsubseteq y},y}   \\
 S4(x = \bagone{} )  \triangleq &\ \   \underset{t\in V^3_{ele}}{\bigwedge}(w_{t,x} = 0)  \\
 S4(x = \bagone{a})  \triangleq &\ \   (w_{a,x} = 1)  \land \underset{t\in V^3_{ele}}{\bigwedge}((a = t) \lor (w_{t,x} = 0))    \\
 S4(x = y \cup z)    \triangleq &\ \   \underset{t\in V^3_{ele}}{\bigwedge}(w_{t,x} = max(w_{t,y},w_{t,z}))   \\
 S4(x = y \cap z)    \triangleq &\ \   \underset{t\in V^3_{ele}}{\bigwedge}(w_{t,x} = min(w_{t,y},w_{t,z}))   \\
 S4(x = y \uplus z)   \triangleq &\ \   \underset{t\in V^3_{ele}}{\bigwedge}(w_{t,x} = w_{t,y}+w_{t,z})   \\
 S4(x = y \setminus z) \triangleq &\ \   \underset{t\in V^3_{ele}}{\bigwedge}(w_{t,x} = \max(0,(w_{t,y}-w_{t,z})))
\end{align*}
\caption{Translation of mixed and bags atoms}
\label{fig:S4}
\end{mdframed}
\end{figure}

\begin{myprop}
\label{prop:final}
The formula $\tilde{F}^4 = S4(\tilde{F}^3)$ is in QFLIA and
%De meme c'est vrai si w = 0 pour tout w par exemple. on doit imposer la creation des w
(i) for any model of $I$ of $\tilde{F}^3$ there is a model $I'$ of $\tilde{F}^4$ such that $I$ and $I'$ agree on $V^3(\int)$.
(ii) for any model $I$ of $\tilde{F}^4$, there is a model $I'$ of $\tilde{F}^3$ such that $I$ and $I'$ agree on $V_{int}(\tilde{F}^3)$.
\end{myprop}

\begin{proof}
Clearly $\tilde{F}^4$ is in QFLIA from the definition of $S4$.

Point (i): Because $S4$ does not rewrite integer atoms, a model $I$ of $\tilde{F}^3$ satisfies, with its integer valuation $I_{int}$, the integer atoms not rewritten by $S4$.
The integer atoms introduced by $S4$ are conform with the semantics of formulas while interpreting the variables $w_{*,x}$ as counting abstractions of a bag $x$.
Then if $I$ does not interpret the variables $w_{*,x}$ as counting abstractions of a bag $x$, we construct $I'$ such that it does and agree on $V^3_{int}$ with $I$.
If $x \neq y$ is satisfied then there exists a value $v$ in $\ZZ$ such as $I_{bag}(x)(v) \neq I_{bag}(y)(v)$. By the choice of $I_{int}$ in the 3rd step (i.e., it satisfies $F_3$), we have that $I_{int}(a_{x\neq y}) = v$
If $m = max(x)$ is satisfied then either $I_{bag}(x) = \bagone{} $ and $I_{ext}(m) = I_{ext}(m_{\bot})$, or $I_{ext}(m)$ is the greatest integer of $\mathcal{D}(I_{bag}(x))$ but different from $I_{ext}(m_{\bot})$ and $I_{ext}(m_{\top})$.

Point (ii):
\begin{enumerate}
\item[A.1] $I$ is a model of $S4(\tilde{F}^3)$
\item[C.1] there is a model $I'$ of $\tilde{F}^3$ that agrees on $V_{int}(\tilde{F}^3)$ with $I$
\end{enumerate}

Proof:
\begin{enumerate}[1.]
\item %[1.]
$I_{int}$ satisfies the $F_3$ component of $\tilde{F}^3$ which is preserved as it is by $S4$.
By the definition of $S4$, $V_{int}(\tilde{F}^3) = V_{int}(S4(\tilde{F}^3))$.

\item %[2.]
We define a valuation of bag variables in $\tilde{F}^3$, $I'_{bag}$ as follows:
%Same issue V_{int}(\tilde{F}^3)
$$
I'_{bag}(x)(v) \triangleq \left\{\begin{array}{ll}
I_{int}(w_{a,x}) & \mbox{ if } v = I_{int}(a) \mbox{ for some }a\in V^3_{int} \\
I_{int}(w_{m,x}) & \mbox{ if } v = I_{ext}(m) \mbox{ for some }m\in V^3_{ext} \\
0 & \mbox{ otherwise}
\end{array}\right.
$$
Then $I'_{bag}$ is a well funded function.

Proof: By the fact that $I_{int}$ satisfies the constraint $F_3$, which means that values of $w_{a,x}$ and $w_{b,x}$ are the same if $I_{int}(a)=I_{int}(b)$.

\item %[3.]
$I'=(I_{int}(\tilde{F}^3),I'_{bag})$ is a model of $S4(\tilde{F}^3)$.

Proof: The proof proceeds by induction on the form of atoms transformed by $S4$. We show only the main points of the induction, the others are done similarly.



\begin{enumerate}[3.1]
\item %[3.1:]
for $L_{mix} ::=\ m = \max(x)$, if $I \models S4(L_{mix})$ then $I'\models L_{mix}$ (and vice versa). \\

\begin{enumerate}[$\mbox{3.1.}1$]
\item %[3.1.1:]
If $I_{bag}(x) \neq \bagone{}$ : \\
\begin{enumerate}[$\mbox{3.1.1.}1$]
\item %[3.1.1.1:]
From definition of $S4(L_{mix})$, $I_{int}(w_{m,x}) \ge 1$ and $I_{ext}(m) \neq I_{ext}(m_{\bot})$ and $I_{ext}(m) \neq I_{ext}(m_{\bot})$.

\item %[3.1.1.2:]
From 3.1.1.1 and the definition of $I'_{bag}$, $I_{ext}(m) \in \mathcal{D}(I_{bag}(x))$.

\item %[3.1.1.3:]
There is no value $v$ in $\mathcal{D}(I_{bag}(x))$ greater than $I_{ext}(m)$

   By absurd, suppose that such a value exists, i.e., $v > I_{ext}(m)$ and $v \in \mathcal{D}(I_{bag}(x))$.

   From the definition of $I'_{bag}$, there exists $t\in V^3_{ele}$ such as $I_{int}(t)=v > I_{ext}(m)$ or $I_{ext}(t)=v > I_{ext}(m)$  and $v = I_{int}(w_{t,x}) > 0$.

   This is in contradiction with the $S4(L_{mix})$ conjunct saying that $t\le a \lor w_{t,x}=0$.
\end{enumerate}
\item %[3.1.2:]
If $I_{bag}(x) = \bagone{}$ : \\
\begin{enumerate}[$\mbox{3.1.2.}1$]
\item From definition of $S4(L_{mix})$, $I_{ext}(m) = I_{ext}(m_{\bot})$.
\end{enumerate}
\end{enumerate}
%usefull ? we admit in first step ?
The reverse direction is shown similarly.

Hence, 3.1 is valid.


\item %[3.2:]
for $L_{bag}::=\ x = y\cup z$, if $I \models S4(L_{bag})$ then $I'\models L_{bag}$ (and vice versa). \\

\begin{enumerate}[3.2.1]
\item %[3.1.1:]
From definition of $S4(L_{bag})$,
$$\underset{t\in V^3_{ele}}{\bigwedge}(I_{int}(w_{t,x}) = max(I_{int}(w_{t,y}),I_{int}(w_{t,z}))).$$

\item %[3.1.2:]
From 3.2.1 and the definition of $I'_{bag}$,
$$\underset{t\in V^3_{ele}}{\bigwedge}(I'_{bag}(x)(I_{int}(t)) = max(I'_{bag}(y)(I_{int}(t)),I'_{bag}(z)(I_{int}(t)))).$$
or $$\underset{t\in V^3_{ele}}{\bigwedge}(I'_{bag}(x)(I_{ext}(t)) = max(I'_{bag}(y)(I_{ext}(t)),I'_{bag}(z)(I_{ext}(t)))).$$

\item %[3.1.3:]
From 3.2.2 and the definition of $I'_{bag}$,
$$\underset{k\in \ZZ}{\bigwedge}(I'_{bag}(x)(k) = max(I'_{bag}(y)(k),I'_{bag}(z)(k))).$$

\item %[3.1.4:]
From 3.2.3 the semantic 2.2 of $T_{bag}$ and $L_{bag}$,
$$I'_{bag}(x) = I'_{bag}(y) \cup I'_{bag}(z)$$.

\end{enumerate}
%usefull ? we admit in first step ?
The reverse direction is shown similarly.

Hence, 3.2 is valid.

\end{enumerate} %% end of  3

\item %[4.]
$I'=(I_{int},I'_{bag})\models S4(\tilde{F}^3)$

Proof: by 1, 2, and 3 above.


\item %[5.]
C.1

Proof: by 4 and the definition of $I'$.

\end{enumerate}
\end{proof}

% \begin{comp}
% The function $S4$ does not conjunct any atom to $\tilde{F}^3$ while rewriting $L_{int}$ atoms
% Moreover $S4$ just replaces atoms such as $a \in x $,$a \notin x $,$a \in^{n} x $,$a \notin^{n} x $ without adding any conjunct.
% For each atom in $L{_bag}$, $L_{mix}$ not in the previous ones, S4 adds $|V_{int}(\tilde{F}^2) \cup V^{31}_{int})|$ conjuncts to $\tilde{F}^3$.
% \end{comp}
