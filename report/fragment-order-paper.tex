%!TEX root = paper.tex

\section{Decision procedure by reduction to \textsf{QFUFLIA}}
\label{sec:forder}

%The previous section described a decision procedure for a formula in \QFBILIA\ in order to obtain an equi-satisfiable formula in QFLIA.
This section describes a reduction of \QFBILIA\ formulas to an equi-satisfiable formula in quantifier free linear arithmetics with uninterpreted functions QFUFLIA.
The first two steps of the procedure are identical to the ones of the previous procedure, so we only consider a formula $\tilde{F}^2$ as input.
The third step introduces two extremum variables to represent $\bot$ and $\top$, a set of fresh integer variables and a set of uninterpreted function variables to represent
(i) the elements that validates atoms $x\neq y$ and $x\nsubseteq y$ and
(ii) for each multi-set variable $x\in V_{bag}(\tilde{F}^2)$ uninterpreted function variables
$G_{x}$ such as for each integer variable $a\in V_{int}(\tilde{F}^2)$, $G_{x}(a)$ is counting the occurrences of values of $a$ in $x$.
This step only adds to $\tilde{F}^2$ a formula $F_3$ over integer variables and uninterpreted function. The formula $\tilde{F}^2 \land F_3$ is then equi-satisfiable to $F$.
The fourth step removes multi-set terms from $\tilde{F}^3$ using the fresh uninterpreted function variables introduced by the counting abstraction and produces a formula $\tilde{F}^4 \land F_3$ equi-satisfaisable to $F$.


\subsection{QFBILIAUF Syntax}
\label{ssec:synUF}

% We use the classic notations for operations over $\ZZ$.
% Integer constants are denoted by $k$, natural ones by $n$.
% Let $V_{int}=\{a,b,c,\ldots\}$ be a finite set of symbols denoting integer variables, i.e., variables with values in $\ZZ$.
Let $V_{\textit{uf}}=\{G,H,\ldots\}$ be a finite set of symbols denoting uninterpreted function variables, i.e., variables with values in $\ZZ \rightarrow \ZZ$.
% We denote by $\MM[\ZZ]$ the domain of bags over integers, i.e.,the set of functions $\ZZ\flc\NN$.
% Let $V_{bag}=\{x,y,z,\ldots\}$ be a finite set of symbols denoting bag variables, i.e., variables with values in $\MM[\ZZ]$.
We suppose that $V_{int}$, $V_{ext}$, $V_{\textit{uf}}$ and $V_{bag}$ are disjoint and we do not write explicitly the type ($\ZZ$, $\ZZ^{\prec}$, $(\ZZ \rightarrow \ZZ)$ or $\MM[\ZZ]$) of each variable,
where $V_{int}$, $V_{ext}$ and $V_{bag}$ are defined as in the previous section.

\begin{mydef}
A QFBILIAUF\ formula $F$ is defined by the following grammar:
\begin{small}
\begin{align*}
F ::= &\ L \; | \; F\lor F \; | \; F\land F \; | \; \lnot F \; | \; F\limp F
& \textit{formula}
\\
L ::= &\ L_{int} \; |\; L_{bag} \; |\; L_{mix} \; |\; L_{ext}
& \textit{boolean atom}
\\
L_{int} ::= &\ T_{int} = T_{int} \; |\; T_{int} \neq T_{int} \; |\; T_{int} < T_{int}\; |\;  T_{int} \ge T_{int}
\\
L_{ext} ::= &\ T_{ext} = T_{ext} \; |\; T_{ext} \neq T_{ext} \; |\; T_{ext} < T_{ext}\; |\;  T_{ext} \ge T_{ext}
\\
L_{bag} ::= &\ T_{bag} = T_{bag} \; |\;T_{bag} \neq T_{bag} \; |\;
T_{bag} \subseteq T_{bag} \; |\;T_{bag} \nsubseteq T_{bag} \; | \;
\\
&\ T_{bag} < T_{bag} \; |\;T_{bag} \ge T_{bag}
\\
L_{mix} ::= &\ a \in T_{bag}\; |\;a \notin T_{bag}\; |\;a \in^{n} T_{bag}\; |\;a \notin^{n} T_{bag}
\\
T_{int} ::= &\  k \;|\; a \;|\; T_{\textit{uf}} \;|\;  T_{int} + T_{int} \;|\; T_{int} - T_{int}\;|\;
& \textit{integer term}
\\
&\ \max(T_{int},T_{int})\;|\; \min(T_{int},T_{int}) \;|\; \ite(F,T_{int},T_{int})
\\
T_{ext} ::= &\ k \;|\; m \;|\; \min(T_{bag}) \;|\; \max(T_{bag}) \;|\; \ite(F,T_{ext},T_{ext})
& \textit{extremum term}
\\
T_{\textit{uf}} ::= &\ G(T_{int})
&\textit{UF term}
\\
T_{bag} ::= &\ \bagone{}\;|\;\bagone{a} \;|\;x \;|\; T_{bag} \cup T_{bag} \;|\; T_{bag} \cap T_{bag} \;|\;
& \textit{bag term}
\\
&\ T_{bag} \setminus T_{bag}\;|\; T_{bag} \uplus T_{bag} \;|\; \ite(F,T_{bag},T_{bag})
\\
\end{align*}
\end{small}
\end{mydef}


We denote by by $\cT_{\textit{uf}}$ the set of uninterpreted function terms.
For a formula $F$, we denote by $V_{\textit{uf}}(F)$ the set of uninterpreted function variables used in $F$.

\subsection{QFBILIAUF Semantics}
\label{ssec:semUF}

In addition to valuations introduced in Section~\ref{ssec:seem},
we consider a valuation $I_{\textit{uf}}$ of variables in $V_{\textit{uf}}$ is a function mapping variables in $V_{\textit{uf}}$ to values in $(\ZZ \rightarrow \ZZ)$, i.e.,
$I_{\textit{uf}} \; : \; V_{\textit{uf}} \flc (\ZZ \rightarrow \ZZ)$.
A valuation $I$ of variables in $V_{int}\cup V_{\textit{uf}} \cup V_{bag} \cup V_{ext}$ is a tuple of valuations $(I_{int}, I_{\textit{uf}}, I_{bag}, I_{ext})$.

Given a valuation $I$ and an uninterpreted function term $T_{\textit{uf}}$, the valuation of $T_{\textit{uf}}$ in $\ZZ$,
denoted by $I^{\square}(T_{\textit{uf}})$, is defined as follows:
\begin{align*}
I^{\square}(G(T_{int})) \triangleq&\ I_{\textit{uf}}(G)(I^*(T_{int})) \\
\end{align*}

\begin{mydef}
  A QFBILIAUF formula $F$ is \emph{satisfiable} if there exists a valuation $I$, called also its model, such that $I \models F$.
\end{mydef}

\subsection{Third Step: Introducing the Counting Abstraction}
This step adds to $\tilde{F}^2$ a formula $F_3$ that allow to introduce the uninterpreted functions variable that will replace the multi-set ones. The transformation has three steps:

\begin{enumerate}
\item[$S_u$3.1:] Build the set $V^{31}_{int}$ as a set of fresh variables, one variable for each atom $(x \neq y)$ or $(x \nsubseteq y)$ in $\tilde{F}^2$.
We denote these variables by $a_{x \neq y}$, resp. $a_{x \nsubseteq y}$.
Intuitively, these variables are introduced to be able to express the fact that there is a value on which $x$ and $y$ differ, resp. $x$ has more copies than $y$.
see subsection \ref{subsec:rewritting qfuflia}.

\item[$S_u$3.2:] We introduce two fresh extremum variables $m_{\bot}$ and $m_{\top}$.
Intuitively, these variables are introduced to be able to express $\bot$, resp. $\top$.

\item[$S_u$3.3:] Build the set
$$V_{\textit{uf}}  =
\bigcup_{x\in V_{bag}(\tilde{F}^2)} G_{x},$$
where an uninterpreted variable $G_{x}$ is added for each bag variable $x$ in order to represent the bag $x$.
Let $V^3_{int} = V_{int}(\tilde{F}^2) \cup V^{31}_{int}$.
Let $V^3_{ele} = V^3_{int} \cup V_{ext}(\tilde{F}^2)$.

The formula $\tilde{F}^3$ is built as follows:
\begin{small}
\begin{align}
  \tilde{F}^3 \triangleq &\
  \tilde{F}^2 \land F_{3} \\
  F_{3 } \triangleq &\
  \left(\bigwedge_{a \in V^{3}_{int}}(m_{\bot} < a < m_{\top})\right)
  \\
  & \land
  \left(\bigwedge_{m \in V_{ext}(\tilde{F}^2)}(m_{\top} \ge m \ge m_{\bot})\right)
  \\
  & \land
  %\bigwedge_{x \in V_{bag}(\tilde{F}^2)}\left(\land_{a\in V^3_{int}}w_{a,x} \ge 0\right)  WRONG DU TO w_{w_a,x,x}
  \left(\bigwedge_{t \in V^3_{ele},x \in V_{bag}}(G_{x}(t) \ge 0)\right)
\end{align}
\end{small}
The set of variables of $\tilde{F}^3$ are $V_{int}(\tilde{F}^3)$, resp. $V_{\textit{uf}}$, resp. $V_{bag}(\tilde{F}^2)$.
\end{enumerate}

The following property states that $\tilde{F}^3 \sim_{sat} \tilde{F}^2$:
\begin{myprop}
\label{prop:order:countabs}
(i) any model of $\tilde{F}^3$ is a model of $\tilde{F}^2$ and
(ii) for any model of $\tilde{F}^2$, there is a model of $\tilde{F}^3$ that agrees on $V_{int}(\tilde{F}^2)$ and $V_{bag}(\tilde{F}^2)$.
\end{myprop}



\subsection{Fourth Step: Removing Multi-set Constraints}
\label{subsec:rewritting qfuflia}
This step rewrites the bag atoms and the mixed atoms in order to remove the bag terms as so transform $\tilde{F}^3$ to a formula in QFUFLIA.
The transformation preserves the boolean structure of the initial formulas and the integer atoms.
It is given by the $S4_{\textit{uf}}$ function whose definition is very similar to the one used the the first procedure (Figure~\ref{fig:S4}): all variables $w$ representing multi-sets are replaces by functions calls. We included its detailed definition in the appendix, Figure~\ref{fig:SU4}.

\begin{myprop}
\label{prop:order:final}
The formula $\tilde{F}^4 = S_{u}4(\tilde{F}^3)$ is in QFUFLIA and
%De meme suis pas sure se soit vrai du tout. on doit imposer la creation que l'on souhaite
(i) any model of $\tilde{F}^3$ is a model of $\tilde{F}^4$ and
(ii) for any model of $\tilde{F}^4$, there is a model of $\tilde{F}^3$ that agrees on $V_{int}(\tilde{F}^3)$.
\end{myprop}
