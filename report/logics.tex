%!TEX root = main.tex

\section{Logic \QFBILIA}
\label{sec:logic}

This section presents the logic fragment of \emph{quantifier free logic over bags of integers and linear arithmetics}, \QFBILIA.
We define its syntax, its semantics, and the sub-fragments used in the decision procedures defined in Sections~\ref{sec:fsimple} and \ref{sec:forder}.

\subsection{Syntax}
\label{ssec:syn}

We denote by $\bot$, $\top$ the extremum term such as $\forall k \in \ZZ, k > \bot$, resp. $k < \top$, we have $\bot < \top$ \cite{piskac2010decision}.
We donote by $\ZZ^{\prec}$ the set $\ZZ \cup \{\bot,\top\}$.
We use the classic notations for operations over $\ZZ$.
Integer constants are denoted by $k$, natural ones by $n$.
Let $V_{int}=\{a,b,c,\ldots\}$ be a finite set of symbols denoting integer variables, i.e., variables with values in $\ZZ$.
Let $V_{ext}=\{m,n,p,\ldots\}$ be a finite set of symbols denoting extremum variables, i.e., variables with values in $\ZZ^{\prec}$.
We denote by $\MM[\ZZ]$ the domain of bags over integers, i.e.,
the set of functions $\ZZ\flc\NN$.
Let $V_{bag}=\{x,y,z,\ldots\}$ be a finite set of symbols denoting bag variables, i.e., variables with values in $\MM[\ZZ]$.
We suppose that $V_{int}$, $V_{ext}$ and $V_{bag}$ are disjoint and we don't write explicitly the type ($\ZZ$, $\ZZ^{\prec}$ or $\MM[\ZZ]$) of each variable.

%\td{Contraindre $a\in T_{bag}$ ou amélioré procedure. (STEP 2)}

\begin{mydef}
A \QFBILIA\ formula $F$ is defined by the following grammar:
\begin{align*}
F ::= &\ L \; | \; F\lor F \; | \; F\land F \; | \; \lnot F \; | \; F\limp F
& \textit{formula}
\\
L ::= &\ L_{int} \; |\; L_{bag} \; |\; L_{mix} \; |\; L_{ext}
& \textit{boolean atom}
\\
L_{int} ::= &\ T_{int} = T_{int} \; |\; T_{int} \neq T_{int} \; |\; T_{int} < T_{int}\; |\;  T_{int} \ge T_{int}
\\
L_{ext} ::= &\ T_{ext} = T_{ext} \; |\; T_{ext} \neq T_{ext} \; |\; T_{ext} < T_{ext}\; |\;  T_{ext} \ge T_{ext}
\\
L_{bag} ::= &\ T_{bag} = T_{bag} \; |\;T_{bag} \neq T_{bag} \; |\;
T_{bag} \subseteq T_{bag} \; |\;T_{bag} \nsubseteq T_{bag} \; | \;
\\
&\ T_{bag} < T_{bag} \; |\;T_{bag} \ge T_{bag}
\\
L_{mix} ::= &\ a \in T_{bag}\; |\;a \notin T_{bag}\; |\;a \in^{n} T_{bag}\; |\;a \notin^{n} T_{bag}\;|\;
\\
&\ m \in T_{bag}\; |\;m \notin T_{bag}\; |\;m \in^{n} T_{bag}\; |\;m \notin^{n} T_{bag}
\\
T_{int} ::= &\  k \;|\; a \;|\; T_{int} + T_{int} \;|\; T_{int} - T_{int}\;|\;
& \textit{integer term}
\\
&\ \max(T_{int},T_{int})\;|\; \min(T_{int},T_{int}) \;|\; \ite(F,T_{int},T_{int})
\\
T_{ext} ::= &\ k \;|\; m \;|\; \min(T_{bag}) \;|\; \max(T_{bag}) \;|\; \ite(F,T_{ext},T_{ext})
& \textit{extremum term}
\\
T_{bag} ::= &\ \bagone{}\;|\;\bagone{a} \;|\;\bagone{m} \;|\;x \;|\; T_{bag} \cup T_{bag} \;|\; T_{bag} \cap T_{bag} \;|\;
& \textit{bag term}
\\
&\ T_{bag} \setminus T_{bag}\;|\; T_{bag} \uplus T_{bag} \;|\; \ite(F,T_{bag},T_{bag})
\\
\end{align*}
We denote by $\cF$ the set of formulas in \QFBILIA, by $\cT_{int}$ the set of integer terms, by $\cT_{ext}$ the set of extremum terms, and by $\cT_{bag}$ the set of multi-set terms.
\end{mydef}

For a formula $F$, we denote by $V_{int}(F)$, $V_{ext}(F)$ and $V_{bag}(F)$  the set of integer, resp. extremum, resp. multi-set variables used in $F$.
We denote by $V(F) = V_{int}(F) \cup V_{ext}(F) \cup V_{bag}(F)$ the set of variables used in $F$.
The same notation is overloaded for atoms and terms in \QFBILIA.
For a formula $F$, we denote by $L_{int}(F)$, $L_{bag}(F)$, $L_{ext}(F)$ and $L_{mix}(F)$ the multi-set of integer, resp. multi-set, resp. mix literals in $F$.
We denote by $L(F) = L_{int}(F) \uplus L_{bag}(F) \uplus L_{mix}(F) \uplus L_{ext}(F)$ the multi-set of literals in $F$.



\subsection{Semantics}
\label{ssec:sem}
% \td{Reussir a dire que la min(empyset) non defini et emptyset comp toujours faux}

% \subsubsection{General Semantics}
% \subsubsection{\QFBILIA\ Semantics}

A valuation $I_{int}$ of variables in $V_{int}$ is a function mapping variables in $V_{int}$ to values in $\ZZ$, i.e.,
$I_{int} \; : \; V_{int} \flc \ZZ$.
Similarly, $I_{ext} \; : \; V_{ext} \flc \ZZ^{\prec}$ denotes a valuation of variables in $V_{ext}$ 
and
$I_{bag} \; : \; V_{bag} \flc \MM[\ZZ]$ denotes a valuation of variables in $V_{bag}$.
A valuation $I$ of variables in $V_{int}\cup V_{bag} \cup V_{ext}$ is a tuple of valuations $(I_{int}, I_{bag}, I_{ext})$;
we denote by $I_{int}$, $I_{bag}$ and $I_{ext}$ the first, resp. second, resp. third component of a valuation $I$.
Let $\cI$ be the set of valuations over variables in $V_{int}\cup V_{bag}\cup V_{ext}$.
Two valuations $I$ and $I'$ \emph{agree} on a set of variables $V$ iff $\forall v\in V.\ I(v) = I'(v)$.
For a valuation $I_{bag}$ of variables in $V_{bag}$ and a bag variable $x$ we denote by
$\mathcal{D}(I_{bag}(x)) = \left\{ v \in \ZZ | I_{bag}(x)(v) > 0 \right\}$ the integers occuring in the bag.

The semantics of \QFBILIA\ is defined by
\begin{itemize}
\item a function $\ ^*: \cI\flc \cT_{int}\flc \ZZ$ mapping a valuation and an integer term to an integer value,
\item a function $\ ^{\bullet}: \cI\flc \cT_{ext}\flc \ZZ\cup\{\bot,\top\}$ mapping a valuation and an integer term to an extremum value,
\item a function $\ ^\circ: \cI\flc \cT_{bag}\flc \MM[\ZZ]$ mapping a valuation and a multi-set term to a multi-set value,
\item a relation $\ \models \subseteq \cI\times \cF$ between valuations and formulas.
\end{itemize}
The components above are defined inductively on the syntax of terms and formulas in a mutually recursive way.
This is due to the presence of $\ite$ terms.

%\td{bot top mais compliquer car + - a exclure etc ...}
Given a valuation $I$ and an integer term $T_{int}$, the valuation of $T_{int}$ in $\ZZ$,
denoted by $I^*(T_{int})$, is defined as follows:
\begin{align*}
I^*(k) \triangleq&\ k
\\
I^*(a) \triangleq&\ I_{int}(a)
\\
I^*(T^1_{int} + T^2_{int}) \triangleq&\ I^*(T^1_{int})  + I^*(T^2_{int})
\\
I^*(T^1_{int} - T^2_{int}) \triangleq&\ I^*(T^1_{int})  - I^*(T^2_{int})
\\
I^*(\max(T^1_{int}, T^2_{int})) \triangleq&\ \max(I^*(T^1_{int}), I^*(T^2_{int}))
\\
I^*(\min(T^1_{int}, T^2_{int})) \triangleq&\ \min(I^*(T^1_{int}), I^*(T^2_{int}))
% \\
% I^*(\min(T_{bag})) \triangleq&\ k \mbox{ s.t. } I^\circ(T_{bag})(k) > 0 \mbox{ and }\forall k'< k.~I^\circ(T_{bag})(k')=0
% \\
% I^*(\max(T_{bag})) \triangleq&\ k \mbox{ s.t. } I^\circ(T_{bag})(k) > 0 \mbox{ and }\forall k'> k.~I^\circ(T_{bag})(k')=0
\\
I^*(\ite(F, T^1_{int}, T^2_{int})) \triangleq&\
\left\{\begin{array}{ll}
I^*(T^1_{int}) & \mbox{if } I\models F \\
I^*(T^2_{int}) & \mbox{otherwise}
\end{array}\right.
\end{align*}
where $\max(k_1,k_2)$ and $\min(k_1,k_2)$ are the minimum and maximum operations over $\ZZ$.

Given a valuation $I$ and an extremum term $T_{ext}$, the valuation of $T_{ext}$ in $\ZZ\cup\{\bot,\top\}$,
denoted by $I^{\bullet}(T_{ext})$, is defined as follows:
\begin{align*}
% I^{\bullet}(\bot) \triangleq&\ \bot
% \\
% I^{\bullet}(\top) \triangleq&\ \top
% \\
% I^{\bullet}(\min(T_{bag})) \triangleq&\ k \mbox{ s.t. } I^\circ(T_{bag})(k) > 0 \mbox{ and }\forall k'< k.~I^\circ(T_{bag})(k')=0
% \\
% I^{\bullet}(\max(T_{bag})) \triangleq&\ k \mbox{ s.t. } I^\circ(T_{bag})(k) > 0 \mbox{ and }\forall k'> k.~I^\circ(T_{bag})(k')=0
% \\
I^{\bullet}(k) \triangleq&\ k
\\
I^{\bullet}(m) \triangleq&\ I_{ext}(m)
\\
I^{\bullet}(\min(T_{bag})) \triangleq&\
\left\{\begin{array}{ll}
\top \mbox{ if } \forall k',I^\circ(T_{bag})(k')=0 \\
k \mbox{ s.t. } I^\circ(T_{bag})(k) > 0 \mbox{ and }\forall k'< k.~I^\circ(T_{bag})(k')=0
\end{array}\right.
\\
I^{\bullet}(\max(T_{bag})) \triangleq&\
\left\{\begin{array}{ll}
\bot \mbox{ if } \forall k',I^\circ(T_{bag})(k')=0 \\
k \mbox{ s.t. } I^\circ(T_{bag})(k) > 0 \mbox{ and }\forall k'> k.~I^\circ(T_{bag})(k')=0
\end{array}\right.
\\
I^{\bullet}(\ite(F, T^1_{ext}, T^2_{ext})) \triangleq&\
\left\{\begin{array}{ll}
I^{\bullet}(T^1_{ext}) & \mbox{if } I\models F \\
I^{\bullet}(T^2_{ext}) & \mbox{otherwise}
\end{array}\right.
\end{align*}

Given a valuation $I$ and a bag term $T_{bag}$, the valuation of $T_{bag}$ in $\MM[\ZZ]$,
denoted by $I^\circ(T_{bag})$, is defined as follows:
\begin{align*}
I^\circ(\bagone{}) \triangleq&\ M \mbox{ s.t. }\forall k.~M(k)=0
\\
I^\circ(\bagone{a}) \triangleq&\ M \mbox{ s.t. }M(a)=1 \mbox{ and } \forall k\neq a.~M(k)=0
\\
I^\circ(\bagone{m}) \triangleq&\
\left\{\begin{array}{ll}
M \mbox{ s.t. } M(m)=1 \mbox{ and} & \forall k\neq m.~M(k)=0 \\
  & \mbox{if } I_{ext}(m) \not\in\{\bot,\top\} \\
M \mbox{ s.t. } \forall k.~M(k)=0 & \mbox{otherwise}
\end{array}\right.
\\
I^\circ(x) \triangleq&\ I_{bag}(x)
\\
I^\circ(T^1_{bag} \cup T^2_{bag}) \triangleq&\ M\mbox{ s.t. }\forall k.~M(k)=\max(I^\circ(T^1_{bag})(k), I^\circ(T^2_{bag})(k))
\\
I^\circ(T^1_{bag} \cap T^2_{bag}) \triangleq&\ M\mbox{ s.t. }\forall k.~M(k)=\min(I^\circ(T^1_{bag})(k), I^\circ(T^2_{bag})(k))
\\
I^\circ(T^1_{bag} \setminus T^2_{bag}) \triangleq&\ M\mbox{ s.t. }\forall k.~M(k)=\max(0, I^\circ(T^1_{bag})(k) - I^\circ(T^2_{bag})(k))
\\
I^\circ(T^1_{bag} \uplus T^2_{bag}) \triangleq&\ M\mbox{ s.t. }\forall k.~M(k)=I^\circ(T^1_{bag})(k) + I^\circ(T^2_{bag})(k)
\\
I^\circ(\ite(F,T^1_{bag},T^2_{bag})) \triangleq&\
\left\{\begin{array}{ll}
I^\circ(T^1_{bag}) & \mbox{if } I \models F \\
I^\circ(T^2_{bag}) & \mbox{otherwise}
\end{array}\right.
\end{align*}

Given a valuation $I$ and an integer atom $L_{int}$, the satisfiability relation $I \models L_{int}$ is defined by structural induction as follows:
\begin{align*}
I \models T^1_{int}=T^2_{int} \mbox{ iff }& I^*(T^1_{int}) = I^*(T^2_{int}) \\
I \models T^1_{int}\ne T^2_{int} \mbox{ iff }& I^*(T^1_{int}) \ne I^*(T^2_{int}) \\
I \models T^1_{int} < T^2_{int} \mbox{ iff }& I^*(T^1_{int}) < I^*(T^2_{int}) \\
I \models T^1_{int}\ge T^2_{int} \mbox{ iff }& I^*(T^1_{int}) \ge I^*(T^2_{int})
\end{align*}

Given a valuation $I$ and an integer atom $L_{ext}$, the satisfiability relation $I \models L_{ext}$ is defined by structural induction as follows:
\begin{align*}
I \models T^1_{ext}=T^2_{ext} \mbox{ iff }& I^{\bullet}(T^1_{ext}) = I^{\bullet}(T^2_{ext}) \\
I \models T^1_{ext}\ne T^2_{ext} \mbox{ iff }& I^{\bullet}(T^1_{ext}) \ne I^{\bullet}(T^2_{ext}) \\
I \models T^1_{ext} < T^2_{ext} \mbox{ iff }& I^{\bullet}(T^1_{ext}) < I^{\bullet}(T^2_{ext}) \\
I \models T^1_{ext}\ge T^2_{ext} \mbox{ iff }& I^{\bullet}(T^1_{ext}) \ge I^{\bullet}(T^2_{ext})
\end{align*}

Given a valuation $I$ and a bag atom $L_{bag}$, the satisfiability relation $I \models L_{bag}$ is defined by structural induction as follows:
\begin{align*}
I \models T^1_{bag} = T^2_{bag} \mbox{ iff }& \forall k.~I^\circ(T^1_{bag})(k) = I^\circ(T^2_{bag})(k) \\
I \models T^1_{bag} \neq T^2_{bag} \mbox{ iff }& \exists k.~I^\circ(T^1_{bag})(k) \neq I^\circ(T^2_{bag})(k) \\
I \models T^1_{bag} \subseteq T^2_{bag} \mbox{ iff }& \forall k.~I^\circ(T^1_{bag})(k) \leq I^\circ(T^2_{bag})(k) \\
I \models T^1_{bag} \nsubseteq T^2_{bag} \mbox{ iff }& \exists k.~I^\circ(T^1_{bag})(k) > I^\circ(T^2_{bag})(k) \\
I \models T^1_{bag} < T^2_{bag} \mbox{ iff }& I^\circ(\max(T^1_{bag})) < I^\circ(\min(T^2_{bag})) \\
I \models T^1_{bag} \ge T^2_{bag} \mbox{ iff }& I^\circ(\min(T^1_{bag})) \ge I^\circ(\max(T^2_{bag}))
\end{align*}

Given a valuation $I$ and a mixed atom $L_{mix}$, the satisfiability relation $I \models L_{mix}$ is defined by structural induction as follows:
\begin{align*}
I \models a\in T^2_{bag} \mbox{ iff }& I^\circ(T^2_{bag})\big(I_{int}(a)\big) \ge 1 \\
I \models a\notin T^2_{bag} \mbox{ iff }& I^\circ(T^2_{bag})\big(I_{int}(a)\big) = 0 \\
I \models a\in^{n} T^2_{bag} \mbox{ iff }& I^\circ(T^2_{bag})\big(I_{int}(a)\big) \ge n \\
I \models a\notin^{n} T^2_{bag} \mbox{ iff }& I^\circ(T^2_{bag})\big(I_{int}(a)\big) < n \\
I \models m\in T^2_{bag} \mbox{ iff }& \left\{\begin{array}{ll}
I^\circ(T^2_{bag})\big(I_{ext}(m)\big) \ge 1 & \mbox{if } I_{ext}(m)\not\in\{\bot ,\top\} \\
True & \mbox{if } \forall k.~I^\circ(T^2_{bag})\big(k) = 0 \\
     &\quad \land\ (I_{ext}(m) \in \{\bot,\top\})\\
False & \mbox{otherwise}
\end{array}\right.\\
I \models m\notin T^2_{bag} \mbox{ iff }& \left\{\begin{array}{ll}
I^\circ(T^2_{bag})\big(I_{ext}(m)\big) = 0 & \mbox{if } I_{ext}(m)\not\in\{\bot ,\top\} \\
False & \mbox{if } \forall k.~I^\circ(T^2_{bag})\big(k) = 0 \\
     &\quad \land\ (I_{ext}(m) \in \{\bot,\top\})\\
True & \mbox{otherwise}
\end{array}\right.\\
I \models m\in^{n} T^2_{bag} \mbox{ iff }& \left\{\begin{array}{ll}
I^\circ(T^2_{bag})\big(I_{ext}(m)\big) \ge n & \mbox{if } I_{ext}(m)\not\in\{\bot ,\top\} \\
True & \mbox{if } \forall k.~I^\circ(T^2_{bag})\big(k) = 0  \land n = 1\\
     &\quad \land\ (I_{ext}(m) \in \{\bot,\top\})\\
False & \mbox{otherwise}
\end{array}\right.\\
I \models m\notin^{n} T^2_{bag} \mbox{ iff }& \left\{\begin{array}{ll}
I^\circ(T^2_{bag})\big(I_{ext}(m)\big) < n & \mbox{if } I_{ext}(m)\not\in\{\bot ,\top\} \\
False & \mbox{if } \forall k.~I^\circ(T^2_{bag})\big(k) = 0 \land n = 1\\
     &\quad \land\ (I_{ext}(m) \in \{\bot,\top\})\\
True & \mbox{otherwise}
\end{array}\right.\\
\end{align*}

Given a valuation $I$ and a formula $F$, the satisfiability relation $I \models F$ is defined by induction on the form of the formula as follows:
\begin{align*}
I \models L \mbox{ iff }& I \models L \\
I \models F_1\lor F_2 \mbox{ iff } & I \models F_1 \mbox{ or } I\models F_2 \\
I \models F_1\land F_2 \mbox{ iff } & I \models F_1 \mbox{ and } I\models F_2 \\
I \models \lnot F \mbox{ iff } & I \not\models F \\
I \models F_1\limp F_2 \mbox{ iff } & I \models \lnot F_1 \mbox{ or } I\models F_2
\end{align*}

\begin{mydef}
  A \QFBILIA\ formula $F$ is \emph{satisfiable} if there exists a valuation $I$, called also its model, such that $I \models F$.
\end{mydef}

Let denote by $\bagone{F}$ the set of models of $F$. Thus, a formula $F$ is satisfiable if $\bagone{F}\neq\emptyset$.

\begin{mydef}
  Two formula $F$ and $F'$ are \emph{semi-equivalent}, denoted $F \sim F'$,
  if $V(F)\subseteq V(F')$ or $V(F')\subseteq V(F)$ and $\forall I \in \bagone{F},\exists I'\in \bagone{F'}$ that agrees on $V(F)\cap V(F')$, and $\forall I' \in \bagone{F'},\exists I\in \bagone{F}$ that agrees on $V(F)\cap V(F')$.
\end{mydef}

\begin{mydef}
  Two formula $F$ and $F'$ are \emph{equi-satisfiable}, denoted $F \sim_{\mathit{sat}} F'$,
  if $\bagone{F}\neq\emptyset$ if and only if $\bagone{F'}\neq\emptyset$.
\end{mydef}


% \begin{mydef}
%   Two formula $F$ and $F'$ are equivalent, denoted $F \sim F'$,
%   if they have the same set of models, i.e., $\bagone{F}=\bagone{F'}$.
% \end{mydef}

% \subsubsection{\textsf{QFUFLIA} Semantics}
% A valuation $I_{fun}$ of variables in $V_{fun}$ is a function mapping variables in $V_{fun}$ to values in $\ZZ \flc \ZZ$, i.e.,
% $I_{fun} \; : \; V_{fun} \flc (\ZZ\flc\ZZ)$.
% A valuation $I$ of variables in $V_{int}\cup V_{fun}$ is a pair of valuations $(I_{int}, I_{fun})$;
% we denote by $I_{int}$ and $I_{fun}$ the first resp. second component of a valuation $I$.
%
% The semantics of \textsf{QFUFLIA} is defined by
% \begin{itemize}
% \item a function $\ ^*: \cI\flc \cT_{int}\flc \ZZ$ mapping a valuation and an integer term to an integer value,
% \item a function $\ ^*: \cI\flc \cT_{fun}\flc (\ZZ\flc\ZZ)$ mapping a valuation and a multi-set term to a multi-set value,
% \item a relation $\ \models \subseteq \cI\times \cF$ between valuations and formulas.
% \end{itemize}
% %The components above are defined inductively on the syntax of terms and formulas in a mutually recursive way.

% \begin{mydef}
%   A \textsf{QFUFLIA} formula $F$ is \emph{satisfiable} if there exists a valuation $I$, called also its model, such that $I \models F$.
% \end{mydef}

\subsection{Examples}

\begin{myex}
If an integer $a$ is in a multi-set $x$, the multi-set $\llbracket a \rrbracket$ is included in the union of multi-sets $x$ and $y$:
\begin{equation}
\label{ex1}
%(a \notin x) \lor (\llbracket a \rrbracket \subseteq (x\cup y))\notag
(a \in x) \limp (\llbracket a \rrbracket \subseteq (x\cup y))
\end{equation}
\end{myex}

\begin{myex}
If $1$ is the smallest element of a multi-set $x$ then $x$ is less than $y$ 
otherwise $x$ is greater than $z$:
\begin{equation}
\label{ex2}
%\big((\min(x) \neq 1) \lor (x<y)\big) \land \big((\min(x) = 1) \lor (x < z)\big) \notag
\big((\min(x) = 1) \limp (x<y)) \land \big((\min(x) \neq 1) \limp (x < z)\big)
\end{equation}
\end{myex}
\begin{myex}
If the maximum of a multi-set $x$ is $0$ then $x$ is the empty bag or the minimum of $x$ is not $0$:
\begin{equation}
\label{ex3}
%(\max(x) \neq 0) \lor ((x = \bagone{}) \lor (\min(x) \neq 0))
(\max(x) = 0) \limp \big((x = \bagone{}) \lor (\min(x) \neq 0)\big)
\end{equation}
\end{myex}


\subsection{Main Fragments}
\label{ssec:frag}

The \QFBILIA\ logic contains several fragments that are interesting for its reduction to QFLIA theory because
they are simpler in syntax but they are, in general, as expressive as \QFBILIA.

\subsubsection{Fragment $\QFBILIA_\nnf$}
\label{sssec:nnf}

The $\QFBILIA_\nnf$ fragment contains only formulas in the negated normal form, i.e., the logical negation is pushed to the level of literals $L_{int}, L_{bag}, L_{mix}, L_{ext}$ and $\ite$ terms are not present.

\begin{mydef}
\label{def:nnf}
A $\QFBILIA_\nnf$ formula $F$ is defined by the following grammar:
\begin{align*}
F ::= &\ L \; | \; F\lor F \; | \; F\land F
\\
L ::= &\ L_{int} \; |\; L_{bag} \; |\; L_{mix} \; |\; L_{ext}
\\
L_{int} ::= &\ T_{int} = T_{int} \; |\; T_{int} \neq T_{int} \; |\; T_{int} < T_{int}\; |\; T_{int} \ge T_{int}
\\
L_{ext} ::= &\ T_{ext} = T_{ext} \; |\; T_{ext} \neq T_{ext} \; |\; T_{ext} < T_{ext}\; |\;  T_{ext} \ge T_{ext}
\\
L_{bag} ::= &\  T_{bag} = T_{bag} \; |\;T_{bag} \neq T_{bag} \; |\;T_{bag} \subseteq T_{bag} \; |\;T_{bag} \nsubseteq T_{bag} \; | \; \notag\\
&\ T_{bag} < T_{bag} \; |\;T_{bag} \ge T_{bag}
\\
L_{mix}  ::=  &\ T_{int} \in T_{bag}\; |\;T_{int} \notin T_{bag}\; |\;T_{int} \in^{n} T_{bag}\; |\;T_{int} \notin^{n} T_{bag}
\notag\\
T_{int} ::= &\  k \;|\; a \;|\; T_{int} + T_{int} \;|\; T_{int} - T_{int}\;|\;
\\
&\ \max(T_{int},T_{int})\;|\; \min(T_{int},T_{int})
\\
T_{ext} ::= &\ k \;|\; m \;|\; \min(T_{bag}) \;|\; \max(T_{bag})
\\
T_{bag} ::=  &\  \bagone{}\;|\;\bagone{a} \;|\;x \;|\; T_{bag} \cup T_{bag} \;|\; T_{bag} \cap T_{bag}\;|\; T_{bag} \setminus T_{bag}\;|\; T_{bag} \uplus T_{bag}
\end{align*}

We denote by $\cF_\nnf$, $\cT_{int,\nnf}$, $\cT_{ext,\nnf}$, and $\cT_{bag,\nnf}$ the set of formulas, integer terms, resp. extremum terms, resp. multi-set terms  in $\QFBILIA_\nnf$.
\end{mydef}



Thus, the examples (\ref{ex1}), (\ref{ex2}), and (\ref{ex3}) are rewritten in $\QFBILIA_\nnf$ as follows:
\begin{eqnarray}
\label{ex2}
(a \notin x) \lor (\llbracket a \rrbracket \subseteq (x\cup y))
\label{ex1nnf}
\\
\big((\min(x) \neq 1) \lor (x<y)\big) \land \big((\min(x) = 1) \lor (x < z)\big)
\label{ex2nnf}
\\
(\max(x) \neq 0) \lor \big((x = \bagone{}) \lor (\min(x) \neq 0)\big)
\label{ex3nnf}
\end{eqnarray}



Proposition~\ref{prop:nnf} (Section~\ref{ssec:dp-s1}) states that \QFBILIA\ and $\QFBILIA_\nnf$ have the same expressive power, i.e., for any formula $F$ in \QFBILIA, there exists an equivalent formula $\tilde{F}$ in $\QFBILIA_\nnf$.


\subsubsection{Fragment $\QFBILIA_{pure}$}
\label{sssec:pure}

The $\QFBILIA_{pure}$ fragment allows only a restricted syntax for multi-set atoms and terms.
Intuitively, we restrict the syntax to essential terms, where no syntactic sugar can be applied to simplify them.
For example, the multi-set terms of the form $x\cup y \cup z$ are equivalent (semantically speaking)
to a term $x\cup y'$ and the atom $y'=y\cup z$, where $y'$ is a fresh multi-set variable.
Also, the atom $x\leq y$ is equivalent (wrt the semantics, i.e., allows same set of models) to the term $\max(x)\leq \min(y)$.
The integer atoms using multi-set variables are reduced to only two equality constraints between an integer variable and the min or max of some multi-set variable.
Thus, the fragment $\QFBILIA_{pure}$ contains only formulas with the simpler multi-set terms which are furthermore in negative normal form.
%The intuition behind this logic is that we want operation to only occurs between two variables.

%\td{Contraindre $a\in T_{bag}$}
\begin{mydef}
A $\QFBILIA_{pure}$ formula $F$ is defined by the following grammar:
\begin{align*}
F ::= &\ L \; | \; F\lor F \; | \; F\land F
\\
L ::= &\ L_{int} \; |\; L_{bag} \; |\; L_{mix} \; |\; L_{ext}
\\
L_{int} ::= &\ T_{int} = T_{int} \; |\; T_{int} \neq T_{int} \; |\; T_{int} < T_{int}\; |\; T_{int} \ge T_{int}
\\
L_{ext} ::= &\ T_{ext} = T_{ext} \; |\; T_{ext} \neq T_{ext} \; |\; T_{ext} < T_{ext}\; |\; T_{ext} \ge T_{ext}
\\
L_{bag} ::= &\ x = T_{bag} \; |\;x \neq y \; |\;x \subseteq y \; |\;x \nsubseteq y
\\
L_{mix} ::= &\ a \in x\; |\;a \notin x\; |\;a \in^{n} x\; |\;a \notin^{n} x
\\
T_{int} ::= &\ k \;|\; a \;|\; T_{int} + T_{int} \;|\; T_{int} - T_{int}\;|\; \max(T_{int},T_{int})\;|\; \min(T_{int},T_{int})
\\
T_{ext} ::= &\ k \;|\; m \;|\; min(x) \;|\; max(x)
\\
T_{bag} ::= &\ \bagone{}\;|\;\bagone{a} \;|\;x \;|\; x \cup y \;|\; x \cap y\;|\; x \setminus y\;|\; x \uplus y
\end{align*}
\end{mydef}

The examples~(\ref{ex1nnf})--(\ref{ex3nnf}) are semi-equivalent with the following formulas in $\QFBILIA_{pure}$:
\begin{align}
\big((a \notin x) \lor (z_1 \subseteq z_2)\big) \land (\bagone{a} = z_1) \land (z_2 = (x\cup y))
\label{ex1pure}
\\[2mm]
\begin{array}{l}
\big((b_1 \neq 1) \lor (b_2<b_3)\big) \land \big((b_1 = 1) \lor (b_2 < b_4)\big) \\
\land (b_1 = \min(x)) \land (b_2 = \max(x)) \land (b_3 = \min(y)) \land (b_4 = \min(z))
\end{array}
\label{ex2pure}
\\[2mm]
(b_1 \neq 0) \lor \big((x = \bagone{}) \lor (b_1 \neq 0)\big)
\land (b_1 = \min(x))
\label{ex3pure}
\end{align}
We might loose equivalency between formula due to the creation of variable, however semi-equivalency is preserved.

Proposition~\ref{prop:pure} (Section~\ref{ssec:dp-s2}) states that $\QFBILIA_{pure}$ and $\QFBILIA_\nnf$ have the same expressive power.
